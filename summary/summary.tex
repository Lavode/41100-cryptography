\documentclass[a4paper]{scrreprt}

% Uncomment to optimize for double-sided printing.
% \KOMAoptions{twoside}

% Set binding correction manually, if known.
% \KOMAoptions{BCOR=2cm}
% Localization options
\usepackage[english]{babel}
\usepackage[T1]{fontenc}
\usepackage[utf8]{inputenc}

% Quotations
\usepackage{dirtytalk}

% Enhanced verbatim sections. We're mainly interested in
% \verbatiminput though.
\usepackage{verbatim}

% Automatically remove leading whitespace in lstlisting
\usepackage{lstautogobble}

% PDF-compatible landscape mode.
% Makes PDF viewers show the page rotated by 90°.
\usepackage{pdflscape}

% Advanced tables
\usepackage{tabu}
\usepackage{longtable}

% Fancy tablerules
\usepackage{booktabs}

% Graphics
\usepackage{graphicx}

% Current time
\usepackage[useregional=numeric]{datetime2}

% Float barriers.
% Automatically add a FloatBarrier to each \section
\usepackage[section]{placeins}

% Custom header and footer
\usepackage{fancyhdr}

\usepackage{geometry}
\usepackage{layout}

% Math tools
\usepackage{mathtools}
% Math symbols
\usepackage{amsmath,amsfonts,amssymb}
\usepackage{amsthm}

\DeclarePairedDelimiter\abs{\lvert}{\rvert}

% Indistinguishable operator (three stacked tildes)
\newcommand*{\diffeo}{% 
  \mathrel{\vcenter{\offinterlineskip
  \hbox{$\sim$}\vskip-.35ex\hbox{$\sim$}\vskip-.35ex\hbox{$\sim$}}}}

\pagestyle{plain}
% \fancyhf{}
% \lhead{}
% \lfoot{}
% \rfoot{}
% 
% Source code & highlighting
\usepackage{listings}

% Coloured boxes!
\usepackage[most]{tcolorbox}
\newtcolorbox{library}[2][]{
	enhanced,
	sharp corners,
	coltitle=black, % title colour
	colbacktitle=black!10!white, % title bg colour
	halign title=center, % title align
	toptitle=1mm, % Top/bottom additional spacing for title
	bottomtitle=1mm,
	fonttitle=\ttfamily,
	colback=white, % body bg colour
	fontupper=\ttfamily,
	title=#2,#1
}

\newtcolorbox{boxcomment}[2][]{
	enhanced,
	colframe=white, % frame colour
	colbacktitle=white, % title bg colour
	halign=center, % body align
	colback=white, % body bg colour
	fonttitle=\ttfamily,
	fontupper=\ttfamily,
	title=#2,#1
}

% SI units
\usepackage[binary-units=true]{siunitx}
\DeclareSIUnit\cycles{cycles}

% Convenience commands
\newcommand{\mailsubject}{41100 - Cryptography - Summary}
\newcommand{\maillink}[1]{\href{mailto:#1?subject=\mailsubject}
                               {#1}}

% Should use this command wherever the print date is mentioned.
\newcommand{\printdate}{\today}

\subject{41100 - Cryptography}
\title{Summary}

\author{Michael Senn \maillink{michael.senn@students.unibe.ch} - 16-126-880}

\date{\printdate}

% Needs to be the last command in the preamble, for one reason or
% another. 
\usepackage{hyperref}


\begin{document}
\maketitle


\chapter{Introduction}

\section{Goals of cryptography}

\begin{description}
	\item[Confidentiality] Ensure only authorized parties are able to read
		information. Encryption.
	\item[Integrity] Ensure information is not modified en-route.
		Signatures.
	\item[Availability] Ensure information reachable by authorized parties,
		not very relevant in modern contexts (not a cryptography
		topic).
\end{description}

\section{Generic model for a cryptosystem}

Alice wants to send message to Bob. Given a system $\Sigma$:

\begin{library}{$\Sigma$}
	\begin{lstlisting}[mathescape=true,autogobble=true]
		KeyGen() -> k
		Enc(k, m) -> c
		Dec(k, c) -> m'
	\end{lstlisting}
\end{library}

Assuming key exchange happens separately and securely, the goal is to achieve:
\begin{description}
	\item[Completness] Bob obtains $m' = m$
	\item[Security] Eve obtains `no useful' information about $m$
\end{description}

Which should hold for any sensible (excluding arbitrarily high computing power)
adversary Eve.

Modern-day security arguments are used which are not from the base up, but are
based on assumptions such as $N \neq NP$.

\section{Kerkhoff's principle}

The security of a crypto system should not rely on the secrecy of the system's
algorithms. This allows studying, reviewing \& reusing a system in public
without compromising its security.

\section{One-time pad}

Perfectly secure, and just as inconvenient.

\begin{library}{Vernam cipher $\Sigma$}
	\begin{lstlisting}[mathescape=true,autogobble=true]
		KeyGen():
		  k <- $\Sigma^l$
		  return k

		Enc(k, m):
		  return $k \oplus m$

		Dec(k, c):
		  return $k \oplus c$
	\end{lstlisting}
\end{library}

\begin{itemize}
	\item Uniform radom ciphertext
	\item Plaintext perfectly hidden
	\item Ciphertext can be decrypted with any key, only one of which
		yields original plaintext
\end{itemize}

\chapter{Provable security}

Information-theoretically secure (eg OTP): No matter computational power of
adversary, no information is gained. Computationally secure: All information is
there, but information lacks time to derive it.

\section{Formalization of encryption}

\subsection{Correctness}

A cryptosystem $\Sigma$ is secure if:
\[
	P[Dec(k, Enc(k, m)) = m] = 1 \forall k \in K, m \in M
\]

\subsection{Exchangable}

Two libraries $L_1, L_2$ are exchangable (notation $L_1 \equiv L_2$) if for all distinguishing algorithms $A$:
\[
	P[A \diamond L_1 = 1] = P[A \diamond L_2 = 1]
\]

\subsection{Uniform ciphertext}

An encryption scheme $\Sigma$ has uniform ciphertexts if the following two
libraries are indistinguishable.

\begin{tcbraster}[raster columns=2,raster equal height,nobeforeafter,raster column skip=2cm]
	\begin{library}{$L_{\text{OTS\$-Real}}$}
		\begin{lstlisting}[mathescape=true,autogobble=true]
			CTXT(m):
			  k <- KeyGen()
			  c <- Enc(k, m)
			  return c
		\end{lstlisting}
	\end{library}
	\begin{library}{$L_{\text{OTS\$-Rand}}$}
		\begin{lstlisting}[mathescape=true,autogobble=true]
			CTXT(m):
			  c <- C
			  return c
		\end{lstlisting}
	\end{library}
\end{tcbraster}


\subsection{One-time secrecy}

An encryption scheme $\Sigma$ has one-time secrecy if the following two
libraries are indistinguishable.

\begin{tcbraster}[raster columns=2,raster equal height,nobeforeafter,raster column skip=2cm]
	\begin{library}{$L_{\text{OTS-Left}}$}
		\begin{lstlisting}[mathescape=true,autogobble=true]
			Eavesdrop($m_l, m_r$):
			  k <- KeyGen()
			  c <- Enc(k, m_l)
			  return c
		\end{lstlisting}
	\end{library}
	\begin{library}{$L_{\text{OTS-Right}}$}
		\begin{lstlisting}[mathescape=true,autogobble=true]
			Eavesdrop($m_l, m_r$):
			  k <- KeyGen()
			  c <- Enc(k, m_r)
			  return c
		\end{lstlisting}
	\end{library}
\end{tcbraster}

\subsection{Chaining lemma}

If $L_1 \equiv L_2$, then $A \diamond L_1 \equiv A \diamond L_2 \forall A$

\subsection{Computational security}

Secure against a probabilistic polynomial time attacker in all but negligibly
many cases.

\subsubsection{Negligible probabilities}

A function $f(\lambda)$ is negligible if $\lim f(\lambda) p(\lambda) = 0
\forall p\ \text{polynomial}$.

\subsubsection{Indistinguishability}

Two libraries $L_1, L_2$ are indistinguishable $L_1 \approx L_2$ if $P[A
\diamond L_1 \rightarrow 1] \approx P[A \diamond L_2 \rightarrow 1]$ for every
probabilistic polynomial-time distinguisher $A$

\chapter{Pseudorandom generator PRG}

A PRG is a deterministic function which:
\begin{itemize}
	\item Stretches a short truly random seed to a large sequence of pseudorando bits
	\item Directly gives a stream cipher
	\item Is the most basic tool of cryptography
\end{itemize}

\section{PRG}

A deterministic function $G: \{0, 1\}^\lambda \rightarrow \{0, 1\}^{\lambda +
l}$ which satisfies $L_{\text{PRG-Real}} \equiv L_{\text{PRG-Rand}}$ where:

\begin{tcbraster}[raster columns=2,raster equal height,nobeforeafter,raster column skip=2cm]
	\begin{library}{$L_{\text{PRG-Real}}$}
		\begin{lstlisting}[mathescape=true,autogobble=true]
			Query():
			  $s \leftarrow \{0, 1\}^\lambda$
			  return G(s)
		\end{lstlisting}
	\end{library}
	\begin{library}{$L_{\text{PRG-Rand}}$}
		\begin{lstlisting}[mathescape=true,autogobble=true]
			Query():
			  $r \leftarrow \{0, 1\}^{\lambda + l}$
			  return r
		\end{lstlisting}
	\end{library}
\end{tcbraster}

Note that $|domain(G)| = 2^\lambda \geq |image(G)|$, with equality iff $G$ is injective.

\subsection{Stream cipher using PRG $G$}

\begin{library}{$S$}
	\begin{lstlisting}[mathescape=true,autogobble=true]
		$K = \{0, 1\}^\lambda$
		$M = \{0, 1\}^{\lambda + l}$
		$C = \{0, 1\}^{\lambda + l}$

		KeyGen():
		  $k \leftarrow K$
		  return k

		Enc(k, m):
		  return $G(k) \oplus m$

		Dec(k, c):
		  return $G(k) \oplus c$
	\end{lstlisting}
\end{library}

If $G$ is a secure PRG, then S possesses computational one-time secrecy.

% TODO 2019-10-11 and 2019-10-30

\chapter{Pseudorandom functions}

Given a key in $\{0, 1\}^\lambda$ it provides, for every input in $\{0,
1\}^{in}$, an output in $\{0, 1\}^{out}$, indistinguishable from uniform random
outputs on that same set.

A PRF $F: \{0, 1\}^\lambda \times \{0, 1\}^{in} \rightarrow \{0, 1\}^{out}$ is
a secure PRF if the following two libraries are indistinguishable.

\begin{tcbraster}[raster columns=2,raster equal height,nobeforeafter,raster column skip=2cm]
	\begin{library}{$L^F_{\text{PRF\$-Real}}$}
		\begin{lstlisting}[mathescape=true,autogobble=true]
			$k \rightarrow \{0, 1\}^\lambda$

			LOOKUP($x \in \{0, 1\}^{in}$):
			  return F(k, x)
		\end{lstlisting}
	\end{library}
	\begin{library}{$L^F_{\text{PRF\$-Rand}}$}
		\begin{lstlisting}[mathescape=true,autogobble=true]
			T := {}

			LOOKUP($x \in \{0, 1\}^{in}$):
			  if x not in T:
			    T[x] $\rightarrow \{0, 1\}^{out}$
			  return T[x]
		\end{lstlisting}
	\end{library}
\end{tcbraster}

\chapter{Block ciphers}

\section{Constructing PRG from PRF}

Given a PRF $F: \{0, 1\}^\lambda \times \{0, 1\}^\lambda \rightarrow \{0,
1\}^\lambda$, we can construct a secure PRG $G: \{0, 1\}^\lambda \rightarrow
\{0, 1\}^{k \cdot \lambda}$ as:

\begin{lstlisting}[mathescape=true,autogobble=true]
	G($s \in \{0, 1\}^\lambda$):
	  for i = 1 to k do:
	    $x_j := F(s, j)$
	  return $x_1 || .. || x_k$
\end{lstlisting}

\section{Pseudo-random permutations PRP}

$F: \{0, 1\}^\lambda \times \{0, 1\}^b \rightarrow \{0, 1\}^b$ is a
pseudorandom permutation on $b$-bit strings if:
\begin{itemize}
	\item There exists a PRP $F^{-1}$ such that $F^{-1}(k, F(k, x)) = x \forall x$
	\item The following two libraries are indistinguishable
\end{itemize}

\begin{tcbraster}[raster columns=2,raster equal height,nobeforeafter,raster column skip=2cm]
	\begin{library}{$L^F_{\text{PRP\$-Real}}$}
		\begin{lstlisting}[mathescape=true,autogobble=true]
			$k \rightarrow \{0, 1\}^\lambda$

			LOOKUP(x)
			  return F(k, x)
		\end{lstlisting}
	\end{library}
	\begin{library}{$L^F_{\text{PRP\$-Rand}}$}
		\begin{lstlisting}[mathescape=true,autogobble=true]
			T := {}
			R := {}

			LOOKUP(x)
			  if x not in T:
			    T[x] = $\{0, 1\}^b \setminus R$
			    $R := R \cup \{T[x]\}$
			  return T[x]
		\end{lstlisting}
	\end{library}
\end{tcbraster}

Indistinguishable as $|R| / 2^\lambda$ negligible.

Further: If F is secure PRP, then if is also secure PRF.

\subsection{Constructing secure PRP from PRF}

Given a PRF $F: \{0, 1\}^\lambda \times \{0, 1\}^\lambda \rightarrow \{0,
1\}^\lambda$, construct a PRP $P$ as follows:

\begin{lstlisting}[mathescape=true,autogobble=true]
	P(k, x || y):
	  return $y || (F(k, y) \oplus x)$

	P^{-1}(k, y || x'):
	  return $(x \oplus F(k, y)) || y$
\end{lstlisting}

\chapter{Security against chosen-plaintext attacks}

Different types of attacks differ by capabilities and/or knowledge of attacker.

\begin{itemize}
	\item Chosen-Plaintext(CPA): Plaintext influenced, ciphertext known
	\item Known-Plaintext: Plaintext and ciphertext known
	\item Ciphertext-only: Ciphertext known
	\item Chosen-cihpertext: Ciphertext influenced and sent to decrypter,
		feedback (eg response time, content) observed
\end{itemize}

In public-key cryptography: Increase of adversarial power for KPA, CPA, CCA,
since everyone able to encrypt anyway, but ciphertext normally not chosen
freely.

For symmetric encryption there is no such direct relation, but anything secure
to CCA is also secure to CPA.

\section{Security against CPA}

$\Sigma$ is secure against CPA if the following two libraries are indistinguishable.

\begin{tcbraster}[raster columns=2,raster equal height,nobeforeafter,raster column skip=2cm]
	\begin{library}{$L^\Sigma_{\text{CPA-Left}}$}
		\begin{lstlisting}[mathescape=true,autogobble=true]
			$k \leftarrow \{0, 1\}^\lambda$
			EAVESDROP($m_l, m_r$):
			  if $|m_l| \neq |m_r|$:
			    return nil

			  return $\Sigma.enc(k, m_l)$
		\end{lstlisting}
	\end{library}
	\begin{library}{$L^\Sigma_{\text{CPA-Right}}$}
		\begin{lstlisting}[mathescape=true,autogobble=true]
			$k \leftarrow \{0, 1\}^\lambda$
			EAVESDROP($m_l, m_r$):
			  if $|m_l| \neq |m_r|$:
			    return nil

			  return $\Sigma.enc(k, m_r)$
		\end{lstlisting}
	\end{library}
\end{tcbraster}

Equality of lengths prevents having to pad ciphertext to fixed length.
Encryption must be non-deterministic, else $EAVESDROP(x, x) == EAVESDROP(x, y)$
would be a distingiusher with advantage $1$ due to the correctness requirement.

\section{Making non-deterministic encryption}

\begin{description}
	\item[Stateful] Keep state inside $\Sigma$, eg a counter. Complex
		implementation, needs synchronization between encryption and
		decryption.
	\item[Randomization in encryption] Randomness becomes part of
		ciphertext, increases length.
	\item[Nonce-based encryption] Each call to $\Sigma.enc$ has to supply a
		nonce (number used only once). Pushes responsibility to caller.
\end{description}

\section{Pseudorandom ciphertext `cpa\$'}

$\Sigma$ has pseudorandom ciphertext security against CPA if following two
libraries are indistinguishable.

\begin{tcbraster}[raster columns=2,raster equal height,nobeforeafter,raster column skip=2cm]
	\begin{library}{$L^\Sigma_{\text{CPA\$-Real}}$}
		\begin{lstlisting}[mathescape=true,autogobble=true]
			$k \leftarrow \{0, 1\}^\lambda$
			CTXT(m):
			  return $\Sigma.enc(k, m)$
		\end{lstlisting}
	\end{library}
	\begin{library}{$L^\Sigma_{\text{CPA\$-Rand}}$}
		\begin{lstlisting}[mathescape=true,autogobble=true]
			$k \leftarrow \{0, 1\}^\lambda$
			CTXT(m):
			  # Random ciphertext such that length correlates with
			  # length of message.
			  $r \leftarrow \Sigma.C | |r| ~ |m|$
			  return $r$
		\end{lstlisting}
	\end{library}
\end{tcbraster}

Pseudorandom ciphertext security implies CPA. Inverse does not hold.

\subsection{CPA\$ security from a secure PRF}

If $F: \{0, 1\}^\lambda \times \{0, 1\}^\lambda \rightarrow \{0, 1\}^\lambda$
be a secure PRF. Then:

\begin{library}{$L^\Sigma_{\text{CPA\$-Rand}}$}
	\begin{lstlisting}[mathescape=true,autogobble=true]
		$\Sigma.KeyGen()$:
		  $k \leftarrow \{0, 1\}^\lambda$
		  ret k

		$\Sigma.Enc(k, m)$:
		  $r \leftarrow \{0, 1\}^\lambda$
		  $k' := F(k, r)$
		  $x := k' \oplus m$

		  ret x, r

		$\Sigma.Dec(k, r, c)$:
		  $k' := F(k, r)$
		  $m := c \oplus k'$

		  ret m
	\end{lstlisting}
\end{library}

As $r$ can be revealed since the attacker does not have access to $k$, it is
likely also ok to use a predictable (eg counting) value of $r$.

\subsection{Block ciphers in practice}

Block ciphers are PRPs. Block lengths are usually 128 bits, with various modes
of operation suporting CPA-secure encryption of arbitrary-length inputs, often
with padding.


\chapter{Encryption modes}

\section{Length expansion}

Ciphertext length is correlated with plaintext length, but the need to further
store initialization vectors, or pad the plaintext, requires a length expansion
in anything but the counter mode.

Usually no relevancy in communication as those use padding, checksums etc
anyway, as well as providing support for transparent packet fragmentation.
Relevancy in storage however, as transparent encryption layer cannot be between
FS and disk, since 1:1 relation of data and written bytes.

\subsection{PKCS\#7 padding}

If n bytes missing for full block, bad with n bytes, each containing the value
of n. If block full, pad with full block to ensure padding recognizable.

\section{ECB}

Encrypt each block independently, ciphertext is concatenation of encrypted
blocks. Insecure as patterns in data still visible.

\section{CBC Cipher Block Chaining}

$c_0$ is IV. Afterwards, $c_i := F_k(m_i \oplus c_{i-1})$. For Decryption $m_i
:= F^{-1}_k(c_i) \oplus c_{i - 1}$.

\section{CTR Counter}

$c_0$ is IV of counter $cnt$. $c_i := F_k(cnt) \oplus m_i$, with $cnt$
increased after every encryption operation.

\section{OFB Output Feedback}

$c_0$ IV. $c_i := F_k(x_i) \oplus m_i$, with $x_i$ being IV for first
operation, and $x_i := F_k(x_{i-1}$ otherwise.


\chapter{Diffie Hellman key agreement}

\section{Modular arithmetic}

\subsection{Integer division}

For $a, d$ rational numbers, there exists quotient $q$, remainder $r$ such that
$a = d * q + r$, where $0 \leq r \leq |d - 1|$.

\subsection{Congruence relation}

$a \equiv b (mod m)$ or $a \equiv_{m} b \Leftrightarrow m | (a - b)$

\subsection{Integers mod $m$}

$Z_m := \{0, 1, \cdots, m - 1\}$

\subsection{Group}

A group $<G, *, 1>$ consist of a set $G$, an operation $*$ and a neutral element $1$ such that:
\begin{itemize}
	\item $\forall a, b \in G: a * b \in G$
	\item $\forall a \in G: 1 * a = a * 1 = a$
	\item $\forall a in G: \exists a^{-1} \in G: a * a^{-1} = 1$
	\item The operation is associative
\end{itemize}

\subsubsection{Examples}

\begin{itemize}
	\item $Z_m := <Z_m, +, 0>$
	\item $Z^{*}_p := <\{1, 2, \cdots, p - 1\}, *, 1>$
\end{itemize}

\subsubsection{Cyclic group}

A group $G$ is cyclic if there is some $g \in G$, called generator, such that
$G = \{g^0, g^1, \cdots, ...\}$. Notation: $<g> = G$

\subsubsection{Primitive root}

If $<g>_p = Z^{*}_p$ then g is a primitive root.

Integers mod m: $<g>_m \subset Z^{*}_m$, $<g>_m = \{g^1 mod m, g^2 mod m, \cdots\}$

\subsubsection{Examples of cyclic groups}

\begin{itemize}
	\item $<1>_{11} = \{1\}$, cycle of length 1
	\item $<2>_{11} = \{1, 2, 4, 8, 5, 10, 9, 7, 3, 6\}$, cycle of length 10
\end{itemize}

\subsubsection{Order of a group}

The order of group $G$ $|G|$ is the number of elements contained within.

\section{Discrete logarithms}

In a cyclic group $G$, the discrete log of $y$ in $G$ with respect to a
primitive root $g$ is defined as $x \in Z_{|G|} : g^x = y$. This is assumed to
be a hard problem.

\subsection{Discrete logarith problem DLP}

Given $y \in G$, compute $x : g^x = y$. For certain groups the DLP is
conjectured to be computationally hard:

\begin{itemize}
	\item Prime-order subgroups of $Z^{*}_p$, $p$ prime, $|p|$ (length of
		prime?) around $2000$
	\item Groups over points on elliptic curves
\end{itemize}

\section{Diffie Hellman key agreement}

Given $G = <g>$ of order $q$.

\begin{lstlisting}[mathescape=true,autogobble=true]
	Alice					Bob
	$a \leftarrow Z_q$			$b \leftarrow Z_q$
	$X := g^a$				$Y := g^b$
			Exchange $X, Y$
	$Z_A := Y^a$				$Z_B := X^b$
\end{lstlisting}

Then: $Z_A = Z_B = g^{ab}$

\subsection{Security}

Communication is public but authenticated. Eve can listen in on public
information ($X, Y$ and $G$) but not influence the messages.

If Eve can compute DLP then it's not secure, otherwise $Z_A = Z_B$ is a
pseudo-random element from the group, indistinguishable in polynomial time.

\section{Key agreement protocol security}

Protocol $\Pi$ generates a key $k \in \Pi.k$, and a transcript $T \in \{0,
1\}^{*}$.

The key agreement protocol $\Pi$ is secure if the following two libraries are
indistinguishable:


\begin{tcbraster}[raster columns=2,raster equal height,nobeforeafter,raster column skip=2cm]
	\begin{library}{$L^\Pi_{\text{KA-Real}}$}
		\begin{lstlisting}[mathescape=true,autogobble=true]
			QUERY():
			  (k, T) := EXEC($\Pi$)
			  return (k, T)
		\end{lstlisting}
	\end{library}
	\begin{library}{$L^\Pi_{\text{KA-Rand}}$}
		\begin{lstlisting}[mathescape=true,autogobble=true]
			QUERY():
			  # Ensure correct probability distribution of T
			  (k, T) := EXEC($\Pi$)
			  $k^{*} \leftarrow \Pi.K$
			  return ($k^{*}$, T)
		\end{lstlisting}
	\end{library}
\end{tcbraster}

In either case Eve sees $(g^a, g^b, g^c)$. In the case of KA-Real, $c = a*b$,
in the case of KA-Rand $c$ is independent.

This assumption is stronger than not being able to compute the discrete
logarithm or the key.

\chapter{Problems and assumptions in discrete-log type cryptography}

\section{DLP}

Given $Y := g^x, x \in Z_q$, compute $x \in Z_q$.

\section{CDHP (computational DH problem)}

Given $A := g^a, B := g^b, a, b \in Z_q$, compute $C := g^{ab}$. Weaker than
DLP.

\section{DDHP (decisional DH problem)}

Given $A := g^a, B := g^b, C := g^{x ? ab : c}, C' := g^c, a, b, c \in Z_q, x
\in \{0, 1\}$, distinguish $C$ and $C'$.

\section{DH KA protocol}

DH KA has security under DDHP, and hence also under CDHP and DLP.

\chapter{Public key cryptosystems}

Each user has public/private keypair. $O(n)$ keys in network of $n$ users,
compared to $O(n^2)$ in symmetric cryptosystems.

\section{Public key cryptosystem $\Sigma$}

$\Sigma$ consist of $M$ message space, $C$ cipher space, $KeyGen() \rightarrow
(pk, sk)$, $Enc(pk, m) \rightarrow c$, $Dec(sk, c) \rightarrow m'$.

\section{Security against pk-CPA}

$\Sigma$ has security against CPA if following two libraries indistinguishable:

\begin{tcbraster}[raster columns=2,raster equal height,nobeforeafter,raster column skip=2cm]
	\begin{library}{$L^\Sigma_{\text{PK-CPA-L}}$}
		\begin{lstlisting}[mathescape=true,autogobble=true]
			$(pk, sk) \leftarrow \Sigma.KeyGen()$

			GETKEY():
			  return pk

			TEST($m_l, m_r$):
			  return Enc(pk, $m_l$)
		\end{lstlisting}
	\end{library}
	\begin{library}{$L^\Sigma_{\text{PK-CPA-R}}$}
		\begin{lstlisting}[mathescape=true,autogobble=true]
			$(pk, sk) \leftarrow \Sigma.KeyGen()$

			GETKEY():
			  return pk

			TEST($m_l, m_r$):
			  return Enc(pk, $m_r$)
		\end{lstlisting}
	\end{library}
\end{tcbraster}

Encryption must once more be non-deterministic as the attacker can encrypt
messages himself.

\section{Pseudorandom ciphertext against pk-CPA}

\begin{tcbraster}[raster columns=2,raster equal height,nobeforeafter,raster column skip=2cm]
	\begin{library}{$L^\Sigma_{\text{PK-CPA\$-Real}}$}
		\begin{lstlisting}[mathescape=true,autogobble=true]
			$(pk, sk) \leftarrow \Sigma.KeyGen()$

			GETKEY():
			  return pk

			CTXT(m):
			  return Enc(pk, m)
		\end{lstlisting}
	\end{library}
	\begin{library}{$L^\Sigma_{\text{PK-CPA\$-Rand}}$}
		\begin{lstlisting}[mathescape=true,autogobble=true]
			$(pk, sk) \leftarrow \Sigma.KeyGen()$

			GETKEY():
			  return pk

			CTXT(m):
			  # Pick such that lengths correlate
			  $c \leftarrow \Sigma.C$
			  return c
		\end{lstlisting}
	\end{library}
\end{tcbraster}

PK-CPA\$-security implies PK-CPA-security.

\section{ElGamal cryptosystem}

Public-key cryptosystem based on DH.

\begin{library}{ElGamal}
	\begin{lstlisting}[mathescape=true,autogobble=true]
		G = <g>
		|G| = q

		KeyGen():
		  $x \leftarrow Z_q$
		  $Y := g^x$
		  return (Y, x)

		Enc(Y, m):
		  $r \leftarrow Z_q$
		  $R := g^r$
		  $C := m * Y^r$

		  return (R, C)

		Dec(x, (R, C)):
		  $m' := C / R^x$
		  return m'
	\end{lstlisting}
\end{library}

\subsection{Hashed ElGamal}

Using a hash function $H: G \rightarrow \{0, 1\}^k$ as a random oracle.

\begin{library}{Hashed ElGamal}
	\begin{lstlisting}[mathescape=true,autogobble=true]
		M = \{0, 1\}^k
		C = G * \{0, 1\}^k

		KeyGen():
		  $x \leftarrow Z_q$
		  $Y := g^x$
		  return (Y, x)

		Enc(Y, m):
		  $r \leftarrow Z_q$
		  $R := g^r$
		  $C := m \oplus H(Y^r)$

		  return (R, C)

		Dec(x, (R, C)):
		  $h := H(R^x)$
		  $m' := C \oplus h$
		  return m'
	\end{lstlisting}
\end{library}

\subsubsection{$H$ in random oracle model (ROM)}

\begin{lstlisting}[mathescape=true,autogobble=true]
	T := []
	H($x \in \{0, 1\}^{*}):
	  if x not in T:
	    T[x] $\leftarrow \{0, 1\}^k$
	  return T[x]
\end{lstlisting}

\chapter{RSA}

% TODO: RSA basics

RSA is first public-key cryptosystem with signatures. Compare DH: Just key
exchange. ElGamal: Just encryption.

RSA is deterministic, and as such not suitable for direct usage as not
CPA-secure.

Restricting $Z^{*}_m := \{a \in Z_m : gcd(a, m) = 1\}$ allows to then compute
the inverse using the extended euclidian algorithm. $egcd(a, b) = (g, s, t)$
then $1 = s*a + t*m \Leftrightarrow s * a \equiv_m 1$.

$gcd(a, m) = 1$ is required to prevent one factor of $m = pq$ being known,
which would conflict the assumption that $m$ is hard to factor.


\section{Euler ($\Phi$) function}

\[
	\Phi: Z^{+} \rightarrow Z^{+}, m \rightarrow |Z^{*}_m| = |\{a \in Z_m : gcd(a, m) = 1\}|
\]

Ie the number of factors in $Z_m$ which are coprime to m, that is $gcd(a, m) =
1$.

\section{RSA function}

\begin{lstlisting}[mathescape=true,autogobble=true]
	KeyGen():
	  # Lambda on order of 1000 bits these days
	  $p \leftarrow PRIMES(\lambda)$
	  $q \leftarrow PRIMES(\lambda)$

	  $N := pq$

	  # e can also be a fixed 'small' prime (smaller than lambda)
	  $e \leftarrow PRIMES(\lambda)$
	  # That is $e * d \equiv_{\phi(N)} 1$
	  $d := e^{-1} mod \phi(N)$

	  return ((N, e), d)

	Eval((n, e), x):
	  return $x^e mod N$

	Invert(d, y):
	  return $y^d mod N$
\end{lstlisting}


\subsection{RSA security}


\begin{tcbraster}[raster columns=2,raster equal height,nobeforeafter,raster column skip=2cm]
	\begin{library}{$L_{\text{RSA-Real}}$}
		\begin{lstlisting}[mathescape=true,autogobble=true]
			((N, e), d) := KeyGen()
			c := {}

			GETKEY():
			  return (N, e)

			GETRAND():
			  $c \leftarrow Z_N$
			  $C << c$

			  return c

			TEST(x, c):
			  if c not in C:
			    return false

			  if $x^e \equiv_N c$:
			    return true
			  else:
			    return false
		\end{lstlisting}
	\end{library}
	\begin{library}{$L_{\text{RSA-Fake}}$}
		\begin{lstlisting}[mathescape=true,autogobble=true]
			# Identical, except:
			TEST(x, c):
			  return false
		\end{lstlisting}
	\end{library}
\end{tcbraster}

Restricting choices of $c$ prevents adversary from manufacturing one value
based on public key which would allow to distinguish the libraries.

\section{Textbook RSA encryption}

Not CPA-secure as deterministic.

\begin{library}{$L_{\text{RSA}}$}
	\begin{lstlisting}[mathescape=true,autogobble=true]
		$(pk, sk) \leftarrow KeyGen()$
		Enc(pk, m) = Eval(pk, m)
		Dec(sk, c) = Invert(sk, c)
	\end{lstlisting}
\end{library}

\section{Secure encryption with RSA, with random oracle}

\begin{library}{$L_{\text{RSA}}$}
	\begin{lstlisting}[mathescape=true,autogobble=true]
		KeyGen() # as in RSA, pk further contains hash function H

		Encrypt((N, e), m):
		  $r \leftarrow Z_N$
		  $R := Eval((N, e), r)$
		  $C := H(r) \oplus m$
		  return (R, C)

		Decrypt(sk, (R, C)):
		  $r := Invert(sk, R)$
		  $m := H(r) \oplus C$
		  return m
	\end{lstlisting}
\end{library}

Random oracle H used, rather than multiplying directly with random value $r$,
to break possible structures. In practice: OAEP with RSA (optimal asymmetric
encryption padding). Advantage over RO in that only one 2k bit component
(rather than two in RO) needed from which to extract message.

Security holds as:
\begin{itemize}
	\item RSA function only applied to randomly chosen values in $Z_N$
	\item RSA function decoupled from message $m$ using RO $H$
	\item Formal security in book
\end{itemize}

\section{Random self reduction}

RSR shows that the worst case is not more dificult than the average case. RSR
turns an algorithm which solves the average case into one which solves the
worst case with only a constant factor as difference.

Assume $A^{*}(c)$ outputs $x$ such that $TEST(x, c \rightarrow True$ with
probability $\delta$ (constant). Then there is a $B(c)$ which outputs $x$ such
that $TEST(x, c) \rightarrow True$ with probability 1:

\begin{lstlisting}[mathescape=true,autogobble=true]
	B(c):
	  $(N, e) \leftarrow KeyGen()$
	  repeat
	    $r \leftarrow Z_N$
	    $c' := r^e * c$
	    $x' := A^{*}(c')$ # Correct with probability delta
	    $x : x' * r^{-1} mod N$
	  until $x^e \equiv_N c$

	  return x
\end{lstlisting}

RSR is an indication that a problem is good for security, as it implies that it
is secure on average, and not only in the worst case.

\chapter{Digital signatures}

\section{Definition of signature scheme $\Gamma$}

\begin{lstlisting}[mathescape=true,autogobble=true]
	M message space
	$\Sigma$ signature space

	KeyGen() -> (pk, sk)

	Sign(sk, m) -> $\sigma$

	Ver(pk, m, $\sigma$) -> True/False
\end{lstlisting}

\section{Security of signature scheme}

Intuition: Secure if adversary not able to fabricate a valid message /
signature pair on his own.

\begin{tcbraster}[raster columns=2,raster equal height,nobeforeafter,raster column skip=2cm]
	\begin{library}{$L^\Gamma_{\text{Sig-Real}}$}
		\begin{lstlisting}[mathescape=true,autogobble=true]
			(pk, sk) <- Gamma.KeyGen()

			GETPK():
			  return pk

			GETSIG(m):
			  return Gamma.Sign(sk, m)

			CHECKSIG(m, sigma):
			  return Gamma.Ver(pk, m, sigma)
		\end{lstlisting}
	\end{library}
	\begin{library}{$L^\Gamma_{\text{Sig-Ideal}}$}
		\begin{lstlisting}[mathescape=true,autogobble=true]
			(pk, sk) <- Gamma.KeyGen()
			S = {}

			GETPK():
			  return pk

			GETSIG(m):
			  sigma := Gamma.Sign(sk, m)
			  S << (m, sigma)
			  return sigma

			CHECKSIG(m, sigma):
			  return (m, sigma) in S
		\end{lstlisting}
	\end{library}
\end{tcbraster}

\section{RSA signatures}

\subsection{RSA textbook signature (insecure)}

\begin{library}{Textbook RSA signature}
	\begin{lstlisting}[mathescape=true,autogobble=true]
		KeyGen():
		  return RSA.KeyGen()

		Sign(sk, m):
		  return RSA.Invert(sk, m)

		Ver(pk, m, sigma):
		return RSA.Eval(pk, sigma) == m
	\end{lstlisting}
\end{library}

Insecure as malleability of RSA allows forging new signatures.

\begin{lstlisting}[mathescape=true,autogobble=true]
	$m_1, m_2$ messages

	$\sigma_1 := Sign(pk, m_1)$
	$\sigma_2 := Sign(pk, m_2)$

	It holds:
	$\sigma_1^e \equiv_N m_1$
	$\sigma_2^e \equiv_N m_2$

	Now:
	$(sigma_1 * sigma_2)^e \equiv_N m_1 * m_2$

	To forge a signature on $m := m_1 * m_2$, compute $\sigma := \sigma_1 * \sigma_2$
\end{lstlisting}

As $m_1, m_2$ can be chosen such that $m = m_1 * m_2$, this breaks the security definition. Worse, signatures can be produced without calling $\Gamma.Sign$: 

\begin{lstlisting}[mathescape=true,autogobble=true]
	Pick random $\sigma'$ in signature space
	$m' := RSA.Eval(pk, \sigma') = (sigma')^e mod N$
	This satisfies $(sigma')^e \equiv_N m'$
	So:
	Ver($pk, m', \sigma'$) = True
\end{lstlisting}

\subsection{RSA full-domain-hash signatures}

Proper signatures with RSA in ROM: Break connection between message $m$ and
signature $\sigma$ by using a hash function $H$, and signing its output.

Let $H: \{0, 1\}^{*} \rightarrow Z_N^{*}$, with $|N| ~ 2048$, $J: \{0, 1\}^{*}
\rightarrow \{0, 1\}^k$, with eg $k = 256$.

Define $H(x) := J(0 || x) || J(1 || x) || \cdots || J(l - 1 || x)$, this way
hash value will have equal length as RSA modulus.

\begin{library}{FDH-RSA}
	\begin{lstlisting}[mathescape=true,autogobble=true]
		KeyGen():
		  # Same as textbook
		  return RSA.KeyGen()

		Sign(sk, m):
		  return RSA.Invert(sk, H(m))

		Ver(pk, m, sigma):
		  return RSA.Eval(pk, sigma) == H(m)
	\end{lstlisting}
\end{library}


\subsection{Schnorr signatures}

Not as popular as eg DSA due to patent issues, PK signatures based on DLP.

\begin{library}{Schnorr signatures}
	\begin{lstlisting}[mathescape=true,autogobble=true]
		$H: {0, 1}^{*} \rightarrow Z_q$

		KeyGen():
		  $x \leftarrow Z_q$
		  return ($g^x, x$)

		Sign(x, m):
		  $r \leftarrow Z_q$
		  $t := g^r mod p$
		  $c := H(m || t)$
		  $s := r - c * x mod q$
		  return (c, s)

		Ver(Y, m, (c, s)):
		  $t' := g^s * Y^c mod p$
		  return $c == H(m || t')$
	\end{lstlisting}
\end{library}

\subsection{DSA}

Standardized `cousin' of Schnorr signatures, found in libraries usually as
ECDSA.

\chapter{Zero knowledge proofs of knowledge}

Goal: Prove knowledge of a secret without revealing it. Applications for
identification protocols, secure computation with robustness against malicious
participants, verifiable encryption (prove that you encrypted something), ...

\section{Interactive proof of knowledge of a discrete logarithm}

Prover $P$, verifier $V$, both known public statement $y$, prover knows secret
$x$ such that $\Psi(y, x) = True$, $V$ will output True/False depending on
whether it accepts the proof.

\subsection{Properties}

\subsubsection{Completness}

If $P$ has input $x$ such that $\Psi(y, x) = True$ then V accepts.

\subsubsection{Soundness}

For a 3-move protocol (3 messages exchanged) of the following form:

\begin{lstlisting}[mathescape=true,autogobble=true]
	1) P -- t --> V
	2) P <- c --- V
	3) P -- s --> V => $\Phi(y, s)$
\end{lstlisting}

There exists an efficient algorithm (`knowledge extractor') that, when given
two transcripts where $V$ accepted $(t, c, s), (t, c', s'), c \neq c'$, can
output $x$ with $\Psi(y, x) = True$.

This is desired (even though it sounds `bad') because it must be possible to
get the knowledge efficiently.


\subsubsection{Zero-knowledge}

V can simulate the transcript with P on its own, without knowing $x$,
generating the same distribution of messages. However it does not have to
generate them in the same order.

Perfect vs computational zero-knowledge: Distribution of messages (simulated vs
real) exchangable vs indistinguishable.

\subsection{ZK-PoK of discrete log}

\begin{lstlisting}[mathescape=true,autogobble=true]
	G = <g>
	|G| = q
	$G \in Z_p$

	P(x, y)				V(y)
	$r \leftarrow Z_q$
	$t := g^r$ 	--->		$t$
	$c$      	<---		$c \leftarrow Z_q$
	$s := r - cx$	--->		$s$
					return $t \equiv_p g^s * y^c$
\end{lstlisting}

$(r, t)$ commitment, $c$ challenge, $s$ response.

\subsubsection{Completness}

\[
	g^r = t = g^{r - cx} * y^c = g^{r - cx} * g^{xc} = g^r
\]

\subsubsection{Soundness}

Given two transcripts $(t, c, s), (t, c', s'), c \neq c'$, such that $V$
accepts both, we can compute $x$.

\subsubsection{Zero-knowledge}

V can generate fake transcript $(T, C, S)$ for given $y$ with the same
distribution:

\begin{lstlisting}[mathescape=true,autogobble=true]
	$C \leftarrow Z_q$
	$S \leftarrow Z_q$
	$T := g^S y^C$
\end{lstlisting}

\subsubsection{Making ZK-PoK non-interactive}

Order of generation of $t, c, s$ was crucial (else simulation possible). Hence order must be retained when removing interaction. Using one-way function (eg hash) on $t$ to determine $c$ allows removing interaction, called the `fiat-shamir' transformation.

\begin{enumerate}
	\item Start from PoK, which is 3-move protocol
	\item Add context (message $m$)
	\item Apply Fiat-Shamir transformation, $c := H(t || m)$
	\item Output self-generated transcript of ZKP
	\item Verify with verification step of ZKP (act as $V$).
\end{enumerate}



\end{document}

