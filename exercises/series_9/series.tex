\documentclass[a4paper]{scrreprt}

% Uncomment to optimize for double-sided printing.
% \KOMAoptions{twoside}

% Set binding correction manually, if known.
% \KOMAoptions{BCOR=2cm}

% Localization options
\usepackage[english]{babel}
\usepackage[T1]{fontenc}
\usepackage[utf8]{inputenc}

% Quotations
\usepackage{dirtytalk}

% Enhanced verbatim sections. We're mainly interested in
% \verbatiminput though.
\usepackage{verbatim}

% Automatically remove leading whitespace in lstlisting
\usepackage{lstautogobble}

% PDF-compatible landscape mode.
% Makes PDF viewers show the page rotated by 90°.
\usepackage{pdflscape}

% Advanced tables
\usepackage{tabu}
\usepackage{longtable}

% Fancy tablerules
\usepackage{booktabs}

% Graphics
\usepackage{graphicx}

% Current time
\usepackage[useregional=numeric]{datetime2}

% Float barriers.
% Automatically add a FloatBarrier to each \section
\usepackage[section]{placeins}

% Custom header and footer
\usepackage{fancyhdr}

\usepackage{geometry}
\usepackage{layout}

% Math tools
\usepackage{mathtools}
% Math symbols
\usepackage{amsmath,amsfonts,amssymb}
\usepackage{amsthm}
% General symbols
\usepackage{stmaryrd}

\DeclarePairedDelimiter\abs{\lvert}{\rvert}

% Indistinguishable operator (three stacked tildes)
\newcommand*{\diffeo}{% 
  \mathrel{\vcenter{\offinterlineskip
  \hbox{$\sim$}\vskip-.35ex\hbox{$\sim$}\vskip-.35ex\hbox{$\sim$}}}}

% Bullet point
\newcommand{\tabitem}{~~\llap{\textbullet}~~}

\pagestyle{plain}
% \fancyhf{}
% \lhead{}
% \lfoot{}
% \rfoot{}
% 
% Source code & highlighting
\usepackage{listings}

% Coloured boxes!
\usepackage[most]{tcolorbox}
\newtcolorbox{library}[2][]{
	enhanced,
	sharp corners,
	coltitle=black, % title colour
	colbacktitle=black!10!white, % title bg colour
	halign title=center, % title align
	toptitle=1mm, % Top/bottom additional spacing for title
	bottomtitle=1mm,
	fonttitle=\ttfamily,
	colback=white, % body bg colour
	fontupper=\ttfamily,
	title=#2,#1
}

\newtcolorbox{boxcomment}[2][]{
	enhanced,
	colframe=white, % frame colour
	colbacktitle=white, % title bg colour
	halign=center, % body align
	colback=white, % body bg colour
	fonttitle=\ttfamily,
	fontupper=\ttfamily,
	title=#2,#1
}

% SI units
\usepackage[binary-units=true]{siunitx}
\DeclareSIUnit\cycles{cycles}

% Convenience commands
\newcommand{\mailsubject}{41100 - Cryptography - Series 9}
\newcommand{\maillink}[1]{\href{mailto:#1?subject=\mailsubject}
                               {#1}}

% Should use this command wherever the print date is mentioned.
\newcommand{\printdate}{\today}

\subject{41100 - Cryptography}
\title{Series 9}

\author{Michael Senn \maillink{michael.senn@students.unibe.ch} - 16-126-880}

\date{\printdate}

% Needs to be the last command in the preamble, for one reason or
% another. 
\usepackage{hyperref}


\begin{document}
\maketitle


\setcounter{chapter}{8}
\chapter{Series 9}

\section{Diffie-Hellman assumptions}

\subsection{Computational Diffie-Hellman assumption}

Consider the following two libraries.

\begin{tcbraster}[raster columns=2,raster equal height,nobeforeafter,raster column skip=2cm]
	\begin{library}{$L^{G}_{\text{CDH-Real}}$}
		\begin{lstlisting}[mathescape=true,autogobble=true]
			$a, b \leftarrow \mathbb{Z}_q$

			SEED():
			  return ($g^a, g^b$)

			CHECK(x):
			  if $x = {(g^a)}^b$:
			    return 1
			  else:
			    return 0
		\end{lstlisting}
	\end{library}
	\begin{library}{$L^{G}_{\text{CDH-Fake}}$}
		\begin{lstlisting}[mathescape=true,autogobble=true]
			$a, b \leftarrow \mathbb{Z}_q$

			SEED():
			  return ($g^a, g^b$)

			CHECK(x):
			  return 0
		\end{lstlisting}
	\end{library}
\end{tcbraster}

The behaviour of the two libraries differs only in the presence of one
conditional expression. Per the bad-event lemma, the advantage of any
polynomial-time distinguisher $A$ is then bounded by the probability of the
differing branch of the conditional expression being entered.

As the CDH is that given $(g^a, g^b)$ a polynomial-time attacker can not compute
$g^{ab}$ with non-negligible probability, the advantage of any such
distinguisher is hence bounded by a negligible probability. As such the two
libraries are indistinguishable.

\subsection{Relation between computational and decisional Diffie-Hellman assumption}

The CDH is a stronger requirement than the DDH, as it requires computing
$g^{ab}$ in polynomial time with non-negligible probability, rather than merely
differentiating the `real' $g^{ab}$ from a `fake' $g^c$ with non-negligible
probability. To illustrate, if $COMPUTE(g^a, g^b)$ is an algorithm which
computes $g^{ab}$ in polynomial time with non-negligible probability $p$, then
the following distinguisher breaks DDH:

\begin{library}{$A$}
	\begin{lstlisting}[mathescape=true,autogobble=true]
		$(g^a, g^b, x) := QUERY()$
		$g^{ab} := COMPUTE(g^a, g^b)$

		if $x = g^{ab}$:
		  return 1
		else:
		  return 0
	\end{lstlisting}
\end{library}

By nature of its construction:
\[
	P[A \diamond L^{G}_{\text{DH-Real}} \rightarrow 1] = p
\]

And:
\[
	P[A \diamond L^{G}_{\text{DH-Rand}} \rightarrow 1] = \frac{p}{q} + \phi
\]
That is $A$ linked to DH-Rand will output $1$ if either $c$ was randomly chosen
such that $g^c = g^{ab}$ in which case $COMPUTE$ calculates the correct result
with probability $p$, or $c$ was one of the other $q-1$ choices and $COMPUTE$
got the `correct' result by chance, a negligible probability $\phi$.

As such:
\begin{align*}
	Bias(A) & = lim_{q \rightarrow \infty}(p - \frac{p}{q} - \phi)\\
	        & = lim_{q \rightarrow \infty}(p) - lim_{q \rightarrow \infty}\left(\frac{p}{q}\right) - lim_{q \rightarrow \infty}\left(\frac{q-1}{q} \cdot \phi\right) \\
		& = lim_{q \rightarrow \infty}(p) - 0 - 0 \tag*{$\phi$ is negligible, $p \leq 1$} \\
		& \neq 0 \tag*{$p$ is not negligible}
\end{align*}

Hence the distinguisher has a non-negligible advantage, therefore breaking
decisional Diffie-Hellman.

\section{Man-in-the-middle attack on DH}

The proposed scenario leads to the three parties being in posession of the
following components of the key, and the resulting keys:
\\

\begin{tabular}{llll}
	\toprule
	& Alice & Eve & Bob \\
	\midrule
	Key components & $a, g^{b'}$ & $a', b', g^a, g^b$ & $b, g^{a'}$ \\
	Keys & $g^{ab'}$ & $g^{ab'}, g^{a'b}$ & $g^{a'b}$ \\
	\bottomrule
\end{tabular}
\\

This shows that effectively two `secure' channels are established, one between
Alice and Eve, the other between Bob and Eve, while to Alice and Bob it looks
as if they were connected via one secure channel. This allows Eve to - as long
as she can maintain her position such that she can intercept \& modify all
communication between Alice and Bob - not only listen to, but also modify, all
communication. This illustrates the lack of authentication provided by DH,
showcasing the need for an additional layer which prevents MITM attacks during
the initial key exchange.

\section{Quadratic residues}

Let $g$ be a primitive root of $\mathbb{Z}^*_p$, that is $<g> =
\mathbb{Z}^*_p$.

\subsection{$g^a \in \mathbb{QR}^*_p \Rightarrow a \text{ even}$}

Let $g^a \in \mathbb{QR}^*_p$ be a quadratic residue.

Assume that $a$ was odd. Then:
\begin{align*}
	\exists\ b = 2b + 1 \Rightarrow g^a = g^{2b + 1} \tag*{As $a$ odd} \\
	\exists\ x : x^2 \equiv_p g^a \tag*{As $a \in \mathbb{QR}^*_p$} \\
	\exists\ k : x = g^k \tag*{As $x \in \mathbb{Z}^*_p$, and $g$ is a generator} \\
	x^2 = g^{2k} = g^a = g^{2b + 1} \tag*{As per their construction} \\
	\Rightarrow 2k = 2b + 1 \lightning \tag*{Which is a contradiction as $2k$ is even, while $2b + 1$ is odd}
\end{align*}

Hence $a$ cannot be odd, so has to be even.

\subsection{$a \text{ even } \Rightarrow g^a \in \mathbb{QR}^*_p$}

Let $a$ be even, $g^a \in \mathbb{Z}^*_p$. Then:
\begin{align*}
	\exists\ b : a = 2b\\
	\Rightarrow g^a = g^{2b} \equiv_p {g^{b}}^{2}
\end{align*}

Hence $g^a$ is a quadratic residue as per the definition.

\end{document}
