\documentclass[a4paper]{scrreprt}

% Uncomment to optimize for double-sided printing.
% \KOMAoptions{twoside}

% Set binding correction manually, if known.
% \KOMAoptions{BCOR=2cm}

% Localization options
\usepackage[english]{babel}
\usepackage[T1]{fontenc}
\usepackage[utf8]{inputenc}

% Quotations
\usepackage{dirtytalk}

% Enhanced verbatim sections. We're mainly interested in
% \verbatiminput though.
\usepackage{verbatim}

% Automatically remove leading whitespace in lstlisting
\usepackage{lstautogobble}

% PDF-compatible landscape mode.
% Makes PDF viewers show the page rotated by 90°.
\usepackage{pdflscape}

% Advanced tables
\usepackage{tabu}
\usepackage{longtable}

% Fancy tablerules
\usepackage{booktabs}

% Graphics
\usepackage{graphicx}

% Current time
\usepackage[useregional=numeric]{datetime2}

% Float barriers.
% Automatically add a FloatBarrier to each \section
\usepackage[section]{placeins}

% Custom header and footer
\usepackage{fancyhdr}

\usepackage{geometry}
\usepackage{layout}

% Math tools
\usepackage{mathtools}
% Math symbols
\usepackage{amsmath,amsfonts,amssymb}
\usepackage{amsthm}
% General symbols
\usepackage{stmaryrd}

\DeclarePairedDelimiter\abs{\lvert}{\rvert}

% Indistinguishable operator (three stacked tildes)
\newcommand*{\diffeo}{% 
  \mathrel{\vcenter{\offinterlineskip
  \hbox{$\sim$}\vskip-.35ex\hbox{$\sim$}\vskip-.35ex\hbox{$\sim$}}}}

% Bullet point
\newcommand{\tabitem}{~~\llap{\textbullet}~~}

\pagestyle{plain}
% \fancyhf{}
% \lhead{}
% \lfoot{}
% \rfoot{}
% 
% Source code & highlighting
\usepackage{listings}

% Coloured boxes!
\usepackage[most]{tcolorbox}
\newtcolorbox{library}[2][]{
	enhanced,
	sharp corners,
	coltitle=black, % title colour
	colbacktitle=black!10!white, % title bg colour
	halign title=center, % title align
	toptitle=1mm, % Top/bottom additional spacing for title
	bottomtitle=1mm,
	fonttitle=\ttfamily,
	colback=white, % body bg colour
	fontupper=\ttfamily,
	title=#2,#1
}

\newtcolorbox{boxcomment}[2][]{
	enhanced,
	colframe=white, % frame colour
	colbacktitle=white, % title bg colour
	halign=center, % body align
	colback=white, % body bg colour
	fonttitle=\ttfamily,
	fontupper=\ttfamily,
	title=#2,#1
}

% SI units
\usepackage[binary-units=true]{siunitx}
\DeclareSIUnit\cycles{cycles}

% Convenience commands
\newcommand{\mailsubject}{41100 - Cryptography - Series 11}
\newcommand{\maillink}[1]{\href{mailto:#1?subject=\mailsubject}
                               {#1}}

% Should use this command wherever the print date is mentioned.
\newcommand{\printdate}{\today}

\subject{41100 - Cryptography}
\title{Series 11}

\author{Michael Senn \maillink{michael.senn@students.unibe.ch} - 16-126-880}

\date{\printdate}

% Needs to be the last command in the preamble, for one reason or
% another. 
\usepackage{hyperref}


\begin{document}
\maketitle


\setcounter{chapter}{10}

\chapter{Series 11}

\section{Computing with encrypted messages}

\subsection{ElGamal}

Let $\oplus, \otimes$ be the respective multiplications. Let $m_1, m_2$ be two
plaintext messages. Then:
\begin{align*}
	Enc(Y, m_1) & = (g^{r_1}, m_1 Y^{r_1}) \\
	Enc(Y, m_2) & = (g^{r_2}, m_2 Y^{r_2})
\end{align*}

This allows computing a valid encryption of $Enc(Y, m_1 m_2$) as:
\begin{align*}
	(g^{r_1} g^{r_2}, m_1 m_2 Y^{r_1} Y^{r_2}) & = (g^{r_1 + r_2}, m_1 m_2 Y^{r_1 + r_2}) \\
						   & = Enc(p_k, m_1 m_2)
\end{align*}

\subsection{RSA}
\label{sec:malleability_rsa}

Let $\oplus, \otimes$ be the respective multiplications. Let $m_1, m_2$ be two
plaintext messages. Then:
\begin{align*}
	Enc((N, e), m_1) & = m_1^e \mod N \\
	Enc((N, e), m_2) & = m_2^e \mod N
\end{align*}

This allows computing a valid encryption of $Enc((N, e), m_1 m_2$) as:
\begin{align*}
	(m_1^e \mod N) (m_2^e \mod N) & = (m_1^e m_2^e) \mod N \\
				      & = (m_1 m_2)^e \mod N \\
				      & = Enc((N, e), m_1 m_2)
\end{align*}


\section{RSA parametes}

\subsection{Odd encryption exponent $e$}

As $p, q$ are chosen as big primes, they are both odd. Hence $\phi(pq) = (p -
1) (q - 1)$ is even. Per definition $e$ and $\phi(pq)$ have to be coprime,
which requires $e$ to be odd as the two numbers would have $2$ as a common
factor otherwise.

\subsection{Factoring $N$ given $N$ and $\phi(N)$}

Given $N := pq$, and $\phi(N) = (p - 1) (q - 1)$ one can factor $N$ by solving
the two equations as follows:
\begin{align*}
	N & = pq \Rightarrow p = \frac{q}{N} \\
	0 & = (p - 1) (q - 1) - \phi(N) \tag*{Per the theorem shown in the lecture} \\
	  & = (\frac{q}{N} - 1) (q - 1) - \phi(N) \tag*{Per the first equation above} \\
	  & = \frac{q^2}{N} + q*(-1 - \frac{1}{N}) + 1 - \phi(N)
\end{align*}

This can then be solved for $q$ as a regular quadratic equation. Either of the
two solutions then yields one of the possible values of $q$, from which $p = N
/ q$ follows.

\begin{align*}
	p & = \frac{1}{2} (- \sqrt{(N - \phi(N) + 1)^2 - 4N} + N - \phi(N) + 1) \\
	q & = \frac{1}{2} (\sqrt{N - \phi(N) + 1)^2 - 4N} + N - \phi(N) + 1)
\end{align*}


\section{Bad choice of prime factors}

\subsection{Small $p$}

Let $p$ be `small', that is $\abs{p} = O(\log(\lambda))$. Note that then $p =
O(2^{\log(\lambda)}) = O(\lambda)$ Consider the following algorithm:

\begin{library}{$FACTOR(N)$}
	\begin{lstlisting}[mathescape=true,autogobble=true]
		$i : = 2$
		while True:
		  if $i | N$:
		    return ($i, N / i$)
		  i++
	\end{lstlisting}
\end{library}

This algorithm will terminate once it reaches $p$, which will be `small'. As $p
= O(\lambda)$, the algorithm's complexity is given as $O(p) = O(O(\lambda)) =
O(\lambda)$, which is polynomial so is efficient.

\subsection{Small difference of $p, q$}

Let $p, q$ be `close' together, that is $\abs{p - q} = O(\log(\lambda))$.
Consider the following algorithm:

\begin{library}{$FACTOR(N)$}
	\begin{lstlisting}[mathescape=true,autogobble=true]
		k := CEIL($\sqrt{N}$)
		i := 0
		while True:
		  if $(k - i) | N$:
		    return ($k - i, N / (k - i)$)
		  elsif $(k + i) | N$:
		    return ($k + i, N / (k + i)$)
		  i++
	\end{lstlisting}
\end{library}

As $p, q$ are close to each other, they will both be close to $\sqrt{N}$. Hence
the above algorithm will have a complexity of $O(\log(\lambda))$ as it'll
terminate once it reaches either of $p$ or $q$, both of which are within
$O(\log(\lambda))$ of each other, and so also of $\sqrt{N}$. As this is
polynomially bounded the algorithm is efficient.

\section{RSA oracle}

Given a ciphertext $c = Enc((N, e), m)$, and the ability to decrypt any
ciphertext $c' \neq c$ we can decrypt $c$ by using the malleability of textbook
RSA. 

Let $c' := Enc((N, e), 2) \cdot c$. Per the malleability shown in subsection
\ref{sec:malleability_rsa}, $c' = Enc((N, e), 2) \cdot Enc((N, e), m)$. Further
$Dec(d, c') = Dec(d, 2) \cdot Dec(d, c) = 2c$, which allows to retrieve $c$.

\end{document}
