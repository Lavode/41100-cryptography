\documentclass[a4paper]{scrreprt}

% Uncomment to optimize for double-sided printing.
% \KOMAoptions{twoside}

% Set binding correction manually, if known.
% \KOMAoptions{BCOR=2cm}

% Localization options
\usepackage[english]{babel}
\usepackage[T1]{fontenc}
\usepackage[utf8]{inputenc}

% Quotations
\usepackage{dirtytalk}

% Enhanced verbatim sections. We're mainly interested in
% \verbatiminput though.
\usepackage{verbatim}

% PDF-compatible landscape mode.
% Makes PDF viewers show the page rotated by 90°.
\usepackage{pdflscape}

% Advanced tables
\usepackage{tabu}
\usepackage{longtable}

% Fancy tablerules
\usepackage{booktabs}

% Graphics
\usepackage{graphicx}

% Current time
\usepackage[useregional=numeric]{datetime2}

% Float barriers.
% Automatically add a FloatBarrier to each \section
\usepackage[section]{placeins}

% Custom header and footer
\usepackage{fancyhdr}

\usepackage{geometry}
\usepackage{layout}

% Math tools
\usepackage{mathtools}
% Math symbols
\usepackage{amsmath,amsfonts,amssymb}
\usepackage{amsthm}

\DeclarePairedDelimiter\abs{\lvert}{\rvert}

\pagestyle{plain}
% \fancyhf{}
% \lhead{}
% \lfoot{}
% \rfoot{}
% 
% Source code & highlighting
\usepackage{listings}

% Convenience commands
\newcommand{\mailsubject}{41100 - Cryptography - Series 1}
\newcommand{\maillink}[1]{\href{mailto:#1?subject=\mailsubject}
                               {#1}}

% Should use this command wherever the print date is mentioned.
\newcommand{\printdate}{\today}

\subject{41100 - Cryptography}
\title{Series 1}

\author{Michael Senn \maillink{michael.senn@students.unibe.ch} - 16-126-880}

\date{\printdate}

% Needs to be the last command in the preamble, for one reason or
% another. 
\usepackage{hyperref}


\begin{document}
\maketitle


\setcounter{chapter}{0}
\chapter{Series 1}

\section{Hiding the plaintext statistics of natural language}

\subsection{Uniform letter statistics and monoalphabetic substitution}

As the compressed text will have a nearly uniform letter statistics, the
monoalphabetic substitution thereof, by extension of being a bijection, will
have so as well. This prevents any cryptanalysis of the cipher text.

As such, each of the $26!$ possible substitution tables is roughly equally
likely, which does require the adversary to systematically try possible
substitution tables, and check whether the decrypted message contains valid
information.

\subsection{Homophonic substitution}

As above, the flattened frequency distribution makes it harder to perform
cryptanalysis.

Furthermore the key space will be significantly larger still, making a
brute-force approach even less realistic. Assuming each letter in the input has
at least one substituent assigned, the theoretical size of the key space is
given as:

\[
	\binom{100}{26} \cdot 26! \cdot 26^{100 - 26}
\]

However many of these keys will be identical, as any permutation of homophones
per plaintext letters yields the same result. A lower bound on the key space
might hence be given as $26^{100 - 26}$, which is still significantly larger
than the $26! < 26^{26} < 26^{74}$ possible keys in the case of monoalphabetic
substitution.

\subsection{Compression and encryption}

If compression is used for the outer layer, that is after encryption, then
compression can, by nature of being an easily reversible process, be undone.
The decompressed ciphertext then offers precisely as much security as the used
encryption algorithm provides.

If a to-be-encrypted message is compressed before encryption, then the effect
is hard to classify. On the one hand compression will hide certain structures
which exist in natural languages, but it might also introduce additional
structure due to eg compression tables or file signatures.

Assuming the used encryption algorithm hides any structure which existed in the
plaintext, the order of operations should have no effect however.

\section{Does the one-time pad leak information?}

Let $m \in \{0, 1\}^\lambda$ be the plaintext message.

Modifying the key space in the way Alice did limits the number of keys to
$2^\lambda - 2$, and will limit the possible cipher texts to $\{0, 1\}^\lambda
\setminus \{m, \neg m\}$. The resulting ciphertext will not be distributed
uniformly over $\{0, 1\}^\lambda$, as two of the possible words have a
probability of zero, while the others have a probability of $\frac{1}{2^\lambda
- 2}$ each.

Therefore, by observing the ciphertext $c$, the adversary knows that the
plaintext is neither $c$ nor $\neg c$, gaining information about the plaintext.
More formally, as $\abs{K} < \abs{M}$, perfect secrecy - ie the cipher text
carrying no information about the plain text - is violated.

Intuitively this can be explained such that, if all keys are allowed and a
cipher text of \say{password = test} is observed, it is equally likely that the
plaintext was \say{password = test}, \say{password = 1234}, or any other
possible text of equal length. Hence allowing keys to be equal to $0^\lambda$
or $1^\lambda$ does not detract from security.

\section{One-time pad using the same key}

Let $m = m_1 \dots m_n \in \{0, 1\}^n$, $m' = m'_1 \dots m'_n \in \{0, 1\}^n$
be two plaintext messages. Let $k = k_1 \dots k_n \in \{0, 1\}^n$ be the shared
key. Let $c = c_1 \dots c_n \in \{0, 1\}^n$, $c' = c'_1 \dots c'_n \in \{0,
1\}^n$ be the two cipher texts.

As $c_i = m_i \oplus k_i$, and $c'_i = m'_i \oplus k_i$, it follows that $c_i =
c'_i \Leftrightarrow m_i = m'_i$. In words: If two corresponding bits of the
cipher texts are equal, the two corresponding bits of the plaintext are equal
as well.

\section{Attack at one-time pad}

\subsection{Known plaintext exposes key}

Let $m = m_1 \dots m_n \in \{0, 1\}^n$ be the plaintext message, $k = k_1 \dots
k_n \in \{0, 1\}^n$ the key. In the case of OTP, the cipher $c = c_1 \dots c_n
\in \{0, 1\}^n$ is then given by $c_i := (m_i + k_i) \mod 2$ or, equivalently,
$c_i := m_i \oplus k_i$

By the properties of the XOR operator $a \oplus b = c \Rightarrow c = a \oplus
b$. Hence each bit of the key can be derived as $k_i := m_i \oplus c_i$.

\subsection{Security of OTP}

Key recovery of OTP is only possible if the plaintext and cipher are known.  As
each key must be used only once, an adversary gains no value from being able to
recover the key used to encrypt a message he already has access to.

\end{document}

