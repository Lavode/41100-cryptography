\documentclass[a4paper]{scrreprt}

% Uncomment to optimize for double-sided printing.
% \KOMAoptions{twoside}

% Set binding correction manually, if known.
% \KOMAoptions{BCOR=2cm}

% Localization options
\usepackage[english]{babel}
\usepackage[T1]{fontenc}
\usepackage[utf8]{inputenc}

% Quotations
\usepackage{dirtytalk}

% Enhanced verbatim sections. We're mainly interested in
% \verbatiminput though.
\usepackage{verbatim}

% PDF-compatible landscape mode.
% Makes PDF viewers show the page rotated by 90°.
\usepackage{pdflscape}

% Advanced tables
\usepackage{tabu}
\usepackage{longtable}

% Fancy tablerules
\usepackage{booktabs}

% Graphics
\usepackage{graphicx}

% Current time
\usepackage[useregional=numeric]{datetime2}

% Float barriers.
% Automatically add a FloatBarrier to each \section
\usepackage[section]{placeins}

% Custom header and footer
\usepackage{fancyhdr}

\usepackage{geometry}
\usepackage{layout}

% Math tools
\usepackage{mathtools}
% Math symbols
\usepackage{amsmath,amsfonts,amssymb}
\usepackage{amsthm}

\DeclarePairedDelimiter\abs{\lvert}{\rvert}

% Indistinguishable operator (three stacked tildes)
\newcommand*{\diffeo}{% 
  \mathrel{\vcenter{\offinterlineskip
  \hbox{$\sim$}\vskip-.35ex\hbox{$\sim$}\vskip-.35ex\hbox{$\sim$}}}}

\pagestyle{plain}
% \fancyhf{}
% \lhead{}
% \lfoot{}
% \rfoot{}
% 
% Source code & highlighting
\usepackage{listings}

% SI units
\usepackage[binary-units=true]{siunitx}
\DeclareSIUnit\cycles{cycles}

% Convenience commands
\newcommand{\mailsubject}{41100 - Cryptography - Series 5}
\newcommand{\maillink}[1]{\href{mailto:#1?subject=\mailsubject}
                               {#1}}

% Should use this command wherever the print date is mentioned.
\newcommand{\printdate}{\today}

\subject{41100 - Cryptography}
\title{Series 5}

\author{Michael Senn \maillink{michael.senn@students.unibe.ch} - 16-126-880}

\date{\printdate}

% Needs to be the last command in the preamble, for one reason or
% another. 
\usepackage{hyperref}


\begin{document}
\maketitle


\setcounter{chapter}{4}
\chapter{Series 5}

\section{A searching adversary}

Let $G_y$ be the image of $G$, that is $G_y := \{G(s) : s \in \{0,
1\}^\lambda\}$.

Note that the distinguisher $A$ effectively iterates through all possible
seeds, checking if at least one of them results in $x$ if used as seed for $G$.
That is $A \rightarrow 1 \Leftrightarrow x \in G_y$.

\subsection{Advantage of distinguisher $A$}

By nature of its construction, $P[A \diamond L^G_{\text{PRG-real}} \rightarrow
1] = 1$.

When linked to $L^G_{\text{PRG-rand}}$, $A$ will return $1$ exactly if $x \in
\{0, 1\}^{\lambda + l}$ is one of the $\abs{G_y} = 2^\lambda$ values produced
by $G$. That is $P[A \diamond L^G_{\text{PRG-rand}} \rightarrow 1] =
\frac{2^\lambda}{2^{\lambda + l}} = 2^{-l}$.

As such the advantage is given as:
\[
	Adv_G = \abs*{P[A \diamond L^G_{\text{PRG-real}} \rightarrow 1] - P[A \diamond L^G_{\text{PRG-rand}} \rightarrow 1]} = 1 - 2^{-l} = \frac{2^l- 1}{2^l}
\]

As $Adv_G \neq 0$, as well as being independent of the security parameter
$\lambda$, $\lim_{\lambda \rightarrow \infty}{Adv_G} \neq 0$, so the advantage
is not negligible.

\subsection{Status of $G$ as PRG}

$G$ is a secure PRG if, for all polynomial distinguishers $A$:

\[
	L^G_{\text{PRG-real}} \diffeo L^G_{\text{PRG-rand}} \Leftrightarrow P[A
	\diamond L^G_{\text{PRG-real}} \rightarrow 1] \approx P[A \diamond
	L^G_{\text{PRG-rand}} \rightarrow 1]
\]

As the given $A$ searches through all $2^\lambda$ seeds, it is exponential
rather than polynomial in $\lambda$. Hence this does not contradict the fact
that $G$ is a secure PRG.

\subsection{Advantage for non-injective $G$}

If $G$ is not injective then $\exists\ s, s' \in \{0, 1\}^\lambda, s \neq s':
G(s) = G(s')$. As such the cardinality of $G_y$ will be lower than if $G$ were
injective.

$P[A \diamond L^G_{\text{PRG-real}} \rightarrow 1]$ remains equal to $1$. $P[A
\diamond L^G_{\text{PRG-rand}} \rightarrow 1]$ will be smaller, as $\abs{G_y} <
2^\lambda$. Hence the advantage of $A$ will be bigger for a non-injective $G$.

\section{Lucky gambler}

As above, let $G_y$ be the image of $G$.

If $A$ is linked to $L^G_{\text{PRG-Real}}$, then $x \in G_y, G(s) \in G_y$. Hence:

\[
	P[A \diamond L^G_{\text{PRG-Real}} \rightarrow 1] = P[x = G(s)] = \frac{1}{\abs{G_y}} = 2^{-\lambda}
\]

If $A$ is linked to $L^G_{\text{PRG-Rand}}$, then $x \in \{0, 1\}^{\lambda + l}, G(s) \in G_y$. Hence:

\begin{align*}
	& P[x \in G_y] = \frac{2^\lambda}{2^{\lambda + l}} = 2^{-l} \\
	& P[A \diamond L^G_{\text{PRG-Rand}} \rightarrow 1 | x \not \in G_y] = 0 \\
	& P[A \diamond L^G_{\text{PRG-Rand}} \rightarrow 1 | x \in G_y] = 2^{-\lambda} \\
	\Rightarrow & P[A \diamond L^G_{\text{PRG-Rand}} \rightarrow 1] = 2^{-l} \cdot 2^{-\lambda} = 2^{-\lambda - l}
\end{align*}

As such the advantage $Adv_G$ is:

\[
	Adv_G = \abs*{P[A \diamond L^G_{\text{PRG-real}} \rightarrow 1] - P[A \diamond L^G_{\text{PRG-rand}} \rightarrow 1]} = 2^{-\lambda - l}
\]

As $l$ is a constant, this is clearly negligible, since:

\[
	\lim_{\lambda \rightarrow \infty}{Adv_G} = \lim_{\lambda \rightarrow \infty}{2^{-\lambda -l}} = 0
\]


\section{PRGs}


\section{PRG candidate}

\end{document}

