\documentclass[a4paper]{scrreprt}

% Uncomment to optimize for double-sided printing.
% \KOMAoptions{twoside}

% Set binding correction manually, if known.
% \KOMAoptions{BCOR=2cm}

% Localization options
\usepackage[english]{babel}
\usepackage[T1]{fontenc}
\usepackage[utf8]{inputenc}

% Quotations
\usepackage{dirtytalk}

% Enhanced verbatim sections. We're mainly interested in
% \verbatiminput though.
\usepackage{verbatim}

% PDF-compatible landscape mode.
% Makes PDF viewers show the page rotated by 90°.
\usepackage{pdflscape}

% Advanced tables
\usepackage{tabu}
\usepackage{longtable}

% Fancy tablerules
\usepackage{booktabs}

% Graphics
\usepackage{graphicx}

% Current time
\usepackage[useregional=numeric]{datetime2}

% Float barriers.
% Automatically add a FloatBarrier to each \section
\usepackage[section]{placeins}

% Custom header and footer
\usepackage{fancyhdr}

\usepackage{geometry}
\usepackage{layout}

% Math tools
\usepackage{mathtools}
% Math symbols
\usepackage{amsmath,amsfonts,amssymb}
\usepackage{amsthm}

\DeclarePairedDelimiter\abs{\lvert}{\rvert}

% Indistinguishable operator (three stacked tildes)
\newcommand*{\diffeo}{% 
  \mathrel{\vcenter{\offinterlineskip
  \hbox{$\sim$}\vskip-.35ex\hbox{$\sim$}\vskip-.35ex\hbox{$\sim$}}}}

\pagestyle{plain}
% \fancyhf{}
% \lhead{}
% \lfoot{}
% \rfoot{}
% 
% Source code & highlighting
\usepackage{listings}

% SI units
\usepackage[binary-units=true]{siunitx}
\DeclareSIUnit\cycles{cycles}

% Convenience commands
\newcommand{\mailsubject}{41100 - Cryptography - Series 5}
\newcommand{\maillink}[1]{\href{mailto:#1?subject=\mailsubject}
                               {#1}}

% Should use this command wherever the print date is mentioned.
\newcommand{\printdate}{\today}

\subject{41100 - Cryptography}
\title{Series 5}

\author{Michael Senn \maillink{michael.senn@students.unibe.ch} - 16-126-880}

\date{\printdate}

% Needs to be the last command in the preamble, for one reason or
% another. 
\usepackage{hyperref}


\begin{document}
\maketitle


\setcounter{chapter}{4}
\chapter{Series 5}

\section{A searching adversary}

Let $G_y$ be the image of $G$, that is $G_y := \{G(s) : s \in \{0,
1\}^\lambda\}$.

Note that the distinguisher $A$ effectively iterates through all possible
seeds, checking if at least one of them results in $x$ if used as seed for $G$.
That is $A \rightarrow 1 \Leftrightarrow x \in G_y$.

\subsection{Advantage of distinguisher $A$}

By nature of its construction, $P[A \diamond L^G_{\text{PRG-real}} \rightarrow
1] = 1$.

When linked to $L^G_{\text{PRG-rand}}$, $A$ will return $1$ exactly if $x \in
\{0, 1\}^{\lambda + l}$ is one of the $\abs{G_y} = 2^\lambda$ values produced
by $G$. That is $P[A \diamond L^G_{\text{PRG-rand}} \rightarrow 1] =
\frac{2^\lambda}{2^{\lambda + l}} = 2^{-l}$.

As such the advantage is given as:
\[
	Adv_G = \abs*{P[A \diamond L^G_{\text{PRG-real}} \rightarrow 1] - P[A \diamond L^G_{\text{PRG-rand}} \rightarrow 1]} = 1 - 2^{-l} = \frac{2^l- 1}{2^l}
\]

As $Adv_G \neq 0$, as well as being independent of the security parameter
$\lambda$, $\lim_{\lambda \rightarrow \infty}{Adv_G} \neq 0$, so the advantage
is not negligible.

\subsection{Status of $G$ as PRG}

$G$ is a secure PRG if, for all polynomial distinguishers $A$:

\[
	L^G_{\text{PRG-real}} \diffeo L^G_{\text{PRG-rand}} \Leftrightarrow P[A
	\diamond L^G_{\text{PRG-real}} \rightarrow 1] \approx P[A \diamond
	L^G_{\text{PRG-rand}} \rightarrow 1]
\]

As the given $A$ searches through all $2^\lambda$ seeds, it is exponential
rather than polynomial in $\lambda$. Hence this does not contradict the fact
that $G$ is a secure PRG.

\subsection{Advantage for non-injective $G$}

If $G$ is not injective then $\exists\ s, s' \in \{0, 1\}^\lambda, s \neq s':
G(s) = G(s')$. As such the cardinality of $G_y$ will be lower than if $G$ were
injective.

$P[A \diamond L^G_{\text{PRG-real}} \rightarrow 1]$ remains equal to $1$. $P[A
\diamond L^G_{\text{PRG-rand}} \rightarrow 1]$ will be smaller, as $\abs{G_y} <
2^\lambda$. Hence the advantage of $A$ will be bigger for a non-injective $G$.

\section{Lucky gambler}

As above, let $G_y$ be the image of $G$.

If $A$ is linked to $L^G_{\text{PRG-Real}}$, then $x \in G_y, G(s) \in G_y$. Hence:

\[
	P[A \diamond L^G_{\text{PRG-Real}} \rightarrow 1] = P[x = G(s)] = \frac{1}{\abs{G_y}} = 2^{-\lambda}
\]

If $A$ is linked to $L^G_{\text{PRG-Rand}}$, then $x \in \{0, 1\}^{\lambda + l}, G(s) \in G_y$. Hence:

\begin{align*}
	& P[x \in G_y] = \frac{2^\lambda}{2^{\lambda + l}} = 2^{-l} \\
	& P[A \diamond L^G_{\text{PRG-Rand}} \rightarrow 1 | x \not \in G_y] = 0 \\
	& P[A \diamond L^G_{\text{PRG-Rand}} \rightarrow 1 | x \in G_y] = 2^{-\lambda} \\
	\Rightarrow & P[A \diamond L^G_{\text{PRG-Rand}} \rightarrow 1] = 2^{-l} \cdot 2^{-\lambda} = 2^{-\lambda - l}
\end{align*}

As such the advantage $Adv_G$ is:

\[
	Adv_G = \abs*{P[A \diamond L^G_{\text{PRG-real}} \rightarrow 1] - P[A \diamond L^G_{\text{PRG-rand}} \rightarrow 1]} = 2^{-l} - 2^{-\lambda - l} = \frac{2^l - 1}{2^{\lambda +l}}
\]

As $l$ is a constant, this is clearly negligible, since:

\[
	\lim_{\lambda \rightarrow \infty}{Adv_G} = \lim_{\lambda \rightarrow \infty}{\frac{2^l - 1}{2^{\lambda +l}}} = \lim_{\lambda \rightarrow \infty}{\frac{1}{2^{\lambda}}} = 0
\]


\section{PRGs}

\subsection{$A(s)$}

Consider the following sequence of operations on the given program.

\begin{lstlisting}[mathescape=true, frame=single]
	A(s):
	  x || y || z := G(s)
	  return G(x) || G(z)
\end{lstlisting}

By definition the seed $s$ has to be chosen uniformally from $\{0,
1\}^\lambda$. Inlining its definition and extracting calls to $\text{QUERY}_G$
as used in claim 5.6 of the book yields the following program linked to
$L^G_{\text{PRG-real}}$:

\begin{lstlisting}[mathescape=true, frame=single]
	A(s):
	  x || y || z := $\text{QUERY}_G$
	  return G(x) || G(z)

	$\diamond$

	$L^G_{\text{PRG-real}}$:
	  $\text{QUERY}_G$:
	    s <- {0, 1}$^\lambda$
	    return G(s)
\end{lstlisting}

As $G$ is a secure PRG $L^G_{\text{PRG-real}}$ can be replaced with $L^G_{\text{PRG-rand}}$:

\begin{lstlisting}[mathescape=true, frame=single]
	A_2(s):
	  x || y || z := $\text{QUERY}_G$
	  return G(x) || G(z)

	$\diamond$

	$L^G_{\text{PRG-rand}}$:
	  $\text{QUERY}_G$:
	    x <- {0, 1}$^{3\lambda}$
	    return x
\end{lstlisting}

Inlining the library yields:

\begin{lstlisting}[mathescape=true, frame=single]
	A_3(s):
	  x <- {0, 1}$^{3\lambda}$
	  y <- {0, 1}$^{3\lambda}$
	  z <- {0, 1}$^{3\lambda}$
	  return x || z
\end{lstlisting}

Finally removing unused variables, and using the fact that choosing $2 x$
uniform bits is equivalent to independently choosing and concatenating $x$
uniform bits twice yields:

\begin{lstlisting}[mathescape=true, frame=single]
	A_4(s):
	  x <- {0, 1}$^{6\lambda}$
	  return x
\end{lstlisting}

Hence we arrived at an indistinguishable implementation of $A$ where the output
is uniformly distributed over $\{0, 1\}^{6 \lambda}$. As such $A$ is a secure
PRG.

\subsection{$B(s)$}

\begin{lstlisting}[mathescape=true, frame=single]
	B(s):
	  x || y || z = G(s)
	  return x || y
\end{lstlisting}

As above we can extract the calls to $G$ into a separate library.

\begin{lstlisting}[mathescape=true, frame=single]
	B_2(s):
	  x || y || z = $\text{QUERY}_G$
	  return x || y

	$\diamond$

	$L^G_{\text{PRG-real}}$:
	  $\text{QUERY}_G$:
	    s <- {0, 1}$^\lambda$
	    return G(s)
\end{lstlisting}

We can then replace $G_{\text{PRG-real}}$ with $G_{\text{PRG-rand}}$, and
inline the call again:

\begin{lstlisting}[mathescape=true, frame=single]
	B_3(s):
	  x || y || z <- {0, 1}$^{3 \lambda}$
	  return x || y
\end{lstlisting}

We then replace the choosing of one $3n$ bit string with the independent
choosing of three $n$ bit strings. Removing the unused variable, and combining
the two concatenated $n$ bit strings into one $2n$ bit string yields the
program below.

\begin{lstlisting}[mathescape=true, frame=single]
	B_4(s):
	  x <- {0, 1}$^{2 \lambda}$
	  return x
\end{lstlisting}

Hence $B$ is a secure PRG as well.

\subsection{$C(s)$}

Consider the following distinguisher:
\begin{lstlisting}[mathescape=true, frame=single]
	A:
	  x || y := QUERY()
	  return x == y
\end{lstlisting}

As per its construction: 
\[
	P[A \diamond L_{\text{PRG-real}} \rightarrow 1] = 1
\]

And, with $x, y$ uniform $3 \lambda$ bit strings:
\[
	P[A \diamond L_{\text{PRG-rand}} \rightarrow 1] = P[x = y] = 2^{-3 \lambda}
\]

Hence the limit of the advantage is:
\[
	\lim_{\lambda \to \infty} Adv_G = \lim_{\lambda \to \infty} 1 - 2^{-3 \lambda} = 1
\]

As such it is not negligible, and $C$ not a secure PRG.

\subsection{$D(s)$}

Consider the following distinguisher:
\begin{lstlisting}[mathescape=true, frame=single]
	A:
	  x || y := QUERY()
	  return y == G($0^\lambda$)
\end{lstlisting}

As per its construction: 
\[
	P[A \diamond L_{\text{PRG-real}} \rightarrow 1] = 1
\]

And, with $x, y$ uniform $3 \lambda$ bit strings:
\[
	P[A \diamond L_{\text{PRG-rand}} \rightarrow 1] = P[y = G(0^\lambda)] = 2^{-3 \lambda}
\]

Hence the limit of the advantage is:
\[
	\lim_{\lambda \to \infty} Adv_G = \lim_{\lambda \to \infty} 1 - 2^{-3 \lambda} = 1
\]

As such it is not negligible, and $C$ not a secure PRG.

\subsection{$E(s)$}

\begin{lstlisting}[mathescape=true, frame=single]
	E(s):
	  x = G(s)
	  y = G($0^\lambda$)
	  return $x \oplus y$
\end{lstlisting}

As $G$ is a secure PRG $x$ is indistinguishable from a uniformly chosen $3
\lambda$ bit string. $y = G(0^\lambda)$ meanwhile is a $3 \lambda$ bit constant.

\begin{lstlisting}[mathescape=true, frame=single]
	E_2(s):
	  x <- {0, 1}$^{3 \lambda}$
	  y = G($0^\lambda$)
	  return $x \oplus y$
\end{lstlisting}

As shown earlier in the lecture, the result of a uniform $n$ bit random
variable XOR-ed with another $n$ bit variable is a uniform $n$ bit random
variable too.

\begin{lstlisting}[mathescape=true, frame=single]
	E_3(s):
	  x <- {0, 1}$^{3 \lambda}$
	  return x
\end{lstlisting}

Hence $E$ is a secure PRG.

\subsection{$F(s_L || s_R)$}

\begin{lstlisting}[mathescape=true, frame=single]
	F($s_L, s_R$):
	  x = G($s_L$)
	  y = G($s_R$)
	  return $x \oplus y$
\end{lstlisting}

As $G$ is a secure PRG, $x$ and $y$ can not be distinguished from uniformly
chosen $3 \lambda$ bit strings:

\begin{lstlisting}[mathescape=true, frame=single]
	F($s_L, s_R$):
	  x <- {0, 1}$^{3 \lambda}$
	  y <- {0, 1}$^{3 \lambda}$
	  return $x \oplus y$
\end{lstlisting}

As above the XOR-ed result of two uniformly distributed $n$ bit strings is
uniformly distributed as well:

\begin{lstlisting}[mathescape=true, frame=single]
	F_2($s_L, s_R$):
	  x <- {0, 1}$^{3 \lambda}$
	  return x
\end{lstlisting}

Hence $F$ is a secure PRG.

\subsection{$H(s_L || s_R)$}

\begin{lstlisting}[mathescape=true, frame=single]
	H($s_L, s_R$):
	  x = G($s_L$)
	  y = G($s_R$)
	  return x || y
\end{lstlisting}

Once more, as $G$ is a secure PRG, $x$ and $y$ can not be distinguished from
uniformly chosen $3 \lambda$ bit strings:

\begin{lstlisting}[mathescape=true, frame=single]
	H($s_L, s_R$):
	  x <- {0, 1}$^{3 \lambda}$
	  y <- {0, 1}$^{3 \lambda}$
	  return x || y
\end{lstlisting}

Choosing and concatenating two uniform $3 \lambda$ bit strings is equivalent to
choosing one uniform $6 \lambda$ bit string:

\begin{lstlisting}[mathescape=true, frame=single]
	H_2($s_L, s_R$):
	  x <- {0, 1}$^{6 \lambda}$
	  return x
\end{lstlisting}

Hence $H$ is a secure PRG.

\section{PRG candidate}

\subsection{$s || f(s)$}

Consider the following distinguisher:

\begin{lstlisting}[mathescape=true, frame=single]
	A:
	  s, x := QUERY()
	  return f(s) == x
\end{lstlisting}

By nature of the construction of the candidate PRG, $P[A \diamond
L_{\text{PRG-real}} \rightarrow 1] = 1$, as the returned pseudo-random variable
fulfills $x = f(s)$.

Further it can be shown that, for two unknown uniformly-distributed values of
$s$ and $x$, $P[f(s) = x] = s^{-l}$ holds for any function $f$. As such:

\[
	P[A \diamond L^G_{\text{PRG-Rand}} \rightarrow 1] = P[x = f(s)] = 2^{-l}
\]

Hence the advantage is:
\[
	Adv_G = \abs*{P[A \diamond L^G_{\text{PRG-real}} \rightarrow 1] - P[A \diamond L^G_{\text{PRG-rand}} \rightarrow 1]} = 1 - 2^{-l}
\]

Since $l$ is a constant, $\lim_{\lambda \rightarrow \infty} Adv_G = 1 - 2^{-l}
> 0$, so is not negligible, hence the candidate PRG is not secure.

\subsection{$P[V(G(s)) = s] > negl(\lambda)$}

Consider the following distinguisher:

\begin{lstlisting}[mathescape=true, frame=single]
	A:
	  x := QUERY()
	  s' := V(x)
	  return G(s') == x
\end{lstlisting}

As per the exercise $P[V(G(s)) = s] > negl(\lambda) \forall s$, hence:
\[
	P[A \diamond L_{\text{PRG-real}} \rightarrow 1] > negl(\lambda)
\]

If $A$ is linked to $L_{\text{PRG-rand}}$, then $s'$ will be some value within
$\{0, 1\}^\lambda$, $x$ a random value within $\{0, 1\}^{\lambda + l}$. For any
$x$ in the image of $G$, $P[G(s') = x] > negl(\lambda)$. For any $x$ not in the
image of $G$, $P[G(s') = x] = 0$. Hence:

\[
	P[A \diamond L_{\text{PRG-rand}} \rightarrow 1] > \frac{2^\lambda}{2^{\lambda + l}} \cdot negl(\lambda) = 2^{-l} \cdot negl(\lambda)
\]

Note that $0 < \lim_{\lambda \rightarrow \infty} negl(\lambda) \leq 1$. As $l >
0$ is constant let $c := 1 - 2^{-l}$ with $0 < c < 1$. The advantage is then
bounded by:

\[
	Adv_G > \abs*{negl(\lambda) - 2^{-l} \cdot negl(\lambda)} = \abs*{negl(\lambda) \cdot c} > 0
\]

And hence the advantage is not negligible:
\[
	\lim_{\lambda \to \infty} Adv_G > \lim_{\lambda \to \infty}(negl(\lambda)) \cdot c > 0
\]

\end{document}


