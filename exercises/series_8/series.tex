\documentclass[a4paper]{scrreprt}

% Uncomment to optimize for double-sided printing.
% \KOMAoptions{twoside}

% Set binding correction manually, if known.
% \KOMAoptions{BCOR=2cm}

% Localization options
\usepackage[english]{babel}
\usepackage[T1]{fontenc}
\usepackage[utf8]{inputenc}

% Quotations
\usepackage{dirtytalk}

% Enhanced verbatim sections. We're mainly interested in
% \verbatiminput though.
\usepackage{verbatim}

% Automatically remove leading whitespace in lstlisting
\usepackage{lstautogobble}

% PDF-compatible landscape mode.
% Makes PDF viewers show the page rotated by 90°.
\usepackage{pdflscape}

% Advanced tables
\usepackage{tabu}
\usepackage{longtable}

% Fancy tablerules
\usepackage{booktabs}

% Graphics
\usepackage{graphicx}

% Current time
\usepackage[useregional=numeric]{datetime2}

% Float barriers.
% Automatically add a FloatBarrier to each \section
\usepackage[section]{placeins}

% Custom header and footer
\usepackage{fancyhdr}

\usepackage{geometry}
\usepackage{layout}

% Math tools
\usepackage{mathtools}
% Math symbols
\usepackage{amsmath,amsfonts,amssymb}
\usepackage{amsthm}

\DeclarePairedDelimiter\abs{\lvert}{\rvert}

% Indistinguishable operator (three stacked tildes)
\newcommand*{\diffeo}{% 
  \mathrel{\vcenter{\offinterlineskip
  \hbox{$\sim$}\vskip-.35ex\hbox{$\sim$}\vskip-.35ex\hbox{$\sim$}}}}

\pagestyle{plain}
% \fancyhf{}
% \lhead{}
% \lfoot{}
% \rfoot{}
% 
% Source code & highlighting
\usepackage{listings}

% Coloured boxes!
\usepackage[most]{tcolorbox}
\newtcolorbox{library}[2][]{
	enhanced,
	sharp corners,
	coltitle=black, % title colour
	colbacktitle=black!10!white, % title bg colour
	halign title=center, % title align
	toptitle=1mm, % Top/bottom additional spacing for title
	bottomtitle=1mm,
	fonttitle=\ttfamily,
	colback=white, % body bg colour
	fontupper=\ttfamily,
	title=#2,#1
}

\newtcolorbox{boxcomment}[2][]{
	enhanced,
	colframe=white, % frame colour
	colbacktitle=white, % title bg colour
	halign=center, % body align
	colback=white, % body bg colour
	fonttitle=\ttfamily,
	fontupper=\ttfamily,
	title=#2,#1
}

% SI units
\usepackage[binary-units=true]{siunitx}
\DeclareSIUnit\cycles{cycles}

% Convenience commands
\newcommand{\mailsubject}{41100 - Cryptography - Series 8}
\newcommand{\maillink}[1]{\href{mailto:#1?subject=\mailsubject}
                               {#1}}

% Should use this command wherever the print date is mentioned.
\newcommand{\printdate}{\today}

\subject{41100 - Cryptography}
\title{Series 8}

\author{Michael Senn \maillink{michael.senn@students.unibe.ch} - 16-126-880}

\date{\printdate}

% Needs to be the last command in the preamble, for one reason or
% another. 
\usepackage{hyperref}


\begin{document}
\maketitle


\setcounter{chapter}{7}
\chapter{Series 8}

\section{CPA-secure encryption?}

Let $\Sigma$ be an encryption scheme with CPS-\$ security. Let $\Sigma'$ be an
encryption scheme with $\Sigma'.Enc(k, m) := 00 || \Sigma.enc(k, m)$.

\subsection{CPA\$ security}

$\Sigma'$ does not posess CPA-\$ security.

\begin{proof}
	Consider the following distinguisher

	\begin{library}{$B$}
		\begin{lstlisting}[mathescape=true,autogobble=true]
			$m \leftarrow \{0, 1\}^l$
			# $c_i$ individual bits of output
			$c_1 || c_2 || c_3 || \cdots || c_l = CTXT(m)$

			return $c_1$ == 0 && $c_2$ == 0
		\end{lstlisting}
	\end{library}

	Clearly by its construction:
	\[
		P[B \diamond L^\Sigma_{\text{CPA\$-real}} \rightarrow 1] = 1
	\]

	Further:
	\[
		P[B \diamond L^\Sigma_{\text{CPA\$-rand}} \rightarrow 1] = P[c_1 = c_2 = 0] = \frac{1}{4}
	\]

	As such:
	\[
		\lim_{\lambda \to \infty} Bias(B) = 1 - \frac{1}{4} = \frac{3}{4} \neq 0
	\]

	Which is not negligible, hence $\Sigma'$ does not possess CPA\$ security.
\end{proof}


\subsection{CPA security}

Consider the following pairs of linked libraries.

Starting with $L^{\Sigma'}_{\text{CPA-Left}}$:
\begin{library}{$L^{\Sigma'}_{\text{CPA-Left}}$}
	\begin{lstlisting}[mathescape=true]
$k \leftarrow \{0, 1\}^\lambda$
EAVESDROP($m_l, m_r$):
  if $\abs{m_l} \neq \abs{m_r}$:
    error
  $c := \Sigma.enc(k, m_l)$
  return 00 || c
	\end{lstlisting}
\end{library}

Extracting $L^{\Sigma}_{\text{CPA-Left}}$:

\begin{tcbraster}[raster columns=2,raster equal height,nobeforeafter,raster column skip=2cm]
	\begin{library}{$L^{\Sigma'}_{\text{CPA-Left}}$}
		\begin{lstlisting}[mathescape=true]
EAVESDROP($m_l, m_r$):
  $c := EAVESDROP'(m_l, m_r)$
  return 00 || c
		\end{lstlisting}
	\end{library}
	\begin{library}{$L^{\Sigma}_{\text{CPA-Left}}$}
		\begin{lstlisting}[mathescape=true]
$k \leftarrow \{0, 1\}^\lambda$
EAVESDROP'($m_l, m_r$):
  if $\abs{m_l} \neq \abs{m_r}$:
    error
  return $\Sigma.enc(k, m_l)$
		\end{lstlisting}
	\end{library}
\end{tcbraster}

Due to the CPA\$ and hence CPA security of $\Sigma$:
\begin{tcbraster}[raster columns=2,raster equal height,nobeforeafter,raster column skip=2cm]
	\begin{library}{$L^{\Sigma'}_{\text{CPA-Left}}$}
		\begin{lstlisting}[mathescape=true]
EAVESDROP($m_l, m_r$):
  $c := EAVESDROP'(m_l, m_r)$
  return 00 || c
		\end{lstlisting}
	\end{library}
	\begin{library}{$L^{\Sigma}_{\text{CPA-Right}}$}
		\begin{lstlisting}[mathescape=true]
$k \leftarrow \{0, 1\}^\lambda$
EAVESDROP'($m_l, m_r$):
  if $\abs{m_l} \neq \abs{m_r}$:
    error
  return $\Sigma.enc(k, m_r)$
		\end{lstlisting}
	\end{library}
\end{tcbraster}

And finally inlining the definition once more:
\begin{library}{$L^{\Sigma'}_{\text{CPA-Right}}$}
	\begin{lstlisting}[mathescape=true]
$k \leftarrow \{0, 1\}^\lambda$
EAVESDROP($m_l, m_r$):
  if $\abs{m_l} \neq \abs{m_r}$:
    error
  $c := \Sigma.enc(k, m_r)$
  return 00 || c
	\end{lstlisting}
\end{library}

As $L^{\Sigma'}_{\text{CPA-Left}} \diffeo L^{\Sigma'}_{\text{CPA-Right}}$,
$\Sigma'$ possesses CPA security.

\section{From a PRP to CPA-secure encryption}

\subsection{a)}

\subsubsection{Decryption algorithm}

\begin{library}{$\Sigma$}
	\begin{lstlisting}[mathescape=true,autogobble=true]
		Dec(k, (r, z)):
		  $c = z \oplus r$
		  $m = F^{-1}(k, c)$
		  return m
	\end{lstlisting}
\end{library}

\subsubsection{CPA security}

Consider the following distinguisher:
\begin{library}{$B$}
	\begin{lstlisting}[mathescape=true,autogobble=true]
		$m_l = 0^\lambda$
		$m_r = 1^\lambda$

		$(r_1, z_1) := EAVESDROP(m_l, m_l)$
		$(r_2, z_2) := EAVESDROP(m_l, m_r)$
		$c_1 := r_1 \oplus z_1$
		$c_2 := r_2 \oplus z_2$

		return $c_1 == c_2$
	\end{lstlisting}
\end{library}

By nature of its construction:
\[
	P[B \diamond L^{\Sigma}_{\text{CPA-Left}} \rightarrow 1] = 1
\]

And, as $F$ has to be invertible so is injective:
\[
	P[B \diamond L^{\Sigma}_{\text{CPA-Right}} \rightarrow 1] = 0
\]

Hence $\lim_{\lambda \to \infty} Bias(B) = 1$ which is not negligible,
so the encryption scheme does not possess CPA-security.

\subsection{b)}

\subsubsection{Decryption algorithm}

\begin{library}{$\Sigma$}
	\begin{lstlisting}[mathescape=true,autogobble=true]
		Dec(k, (s, x)):
		  $r := F^{-1}(k, x)$
		  $m = s \oplus r$
		  return m
	\end{lstlisting}
\end{library}

\subsubsection{CPA security}

Consider the following sequence of hybrid libraries, starting with
$L^{\Sigma}_{\text{CPA\$-Real}}$.

\begin{tcbraster}[raster columns=2,raster equal height,nobeforeafter,raster column skip=2cm]
	\begin{library}{$L^{\Sigma}_{\text{CPA\$-Real}}$}
		\begin{lstlisting}[mathescape=true,autogobble=true]
			$k \leftarrow \{0, 1\}^\lambda$
			CTXT($m$):
			  if $\abs{m_l} \neq \abs{m_r}$:
			    error
			  return $\Sigma.enc(k, m)$
		\end{lstlisting}
	\end{library}
	\begin{library}{$\Sigma$}
		\begin{lstlisting}[mathescape=true,autogobble=true]
			Enc(k, m):
			  $r \leftarrow \{0, 1\}^\lambda$
			  $s := r \oplus m$
			  $x = F(k, r)$
			  return (s, x)
		\end{lstlisting}
	\end{library}
\end{tcbraster}

As, by nature of being a secure PRP, $F$ is also a secure $PRF$, extracting
$L^{F}_{\text{PRF-real}}$:

\begin{tcbraster}[raster columns=3,raster equal height,nobeforeafter,raster column skip=0.2cm]
	\begin{library}{$L^{\Sigma}_{\text{CPA\$-Real}}$}
		\begin{lstlisting}[mathescape=true,autogobble=true]
			CTXT($m$):
			  if $\abs{m_l} \neq \abs{m_r}$:
			    error
			  return $\Sigma.enc(k, m)$
		\end{lstlisting}
	\end{library}
	\begin{library}{$\Sigma$}
		\begin{lstlisting}[mathescape=true,autogobble=true]
			Enc(k, m):
			  $r \leftarrow \{0, 1\}^\lambda$
			  $s := r \oplus m$
			  $x := LOOKUP(r)$
			  return (s, x)
		\end{lstlisting}
	\end{library}
	\begin{library}{$L^{F}_{\text{PRF-real}}$}
		\begin{lstlisting}[mathescape=true,autogobble=true]
			$k \leftarrow \{0, 1\}^\lambda$
			LOOKUP(m):
			  return F(k, x)
		\end{lstlisting}
	\end{library}
\end{tcbraster}

As $F$ is a secure PRF:

\begin{tcbraster}[raster columns=3,raster equal height,nobeforeafter,raster column skip=0.2cm]
	\begin{library}{$L^{\Sigma}_{\text{CPA\$-Real}}$}
		\begin{lstlisting}[mathescape=true,autogobble=true]
			CTXT($m$):
			  if $\abs{m_l} \neq \abs{m_r}$:
			    error
			  return $\Sigma.enc(k, m)$
		\end{lstlisting}
	\end{library}
	\begin{library}{$\Sigma$}
		\begin{lstlisting}[mathescape=true,autogobble=true]
			Enc(k, m):
			  $r \leftarrow \{0, 1\}^\lambda$
			  $s := r \oplus m$
			  $x := LOOKUP(r)$
			  return (s, x)
		\end{lstlisting}
	\end{library}
	\begin{library}{$L^{F}_{\text{PRF-rand}}$}
		\begin{lstlisting}[mathescape=true,autogobble=true]
			T := {}
			LOOKUP(m):
			  if m not in T:
			    $T[m] \leftarrow \{0, 1\}^\lambda$
			  return T[m]
		\end{lstlisting}
	\end{library}
\end{tcbraster}

As a polynomial-time attacker can at most query $LOOKUP()$ polyonomial times,
selecting from $\{0, 1\}^\lambda$ without collision can not be distinguished
from selecting from the same set with collisions, as demonstrated in the
lecture. Inlining that code then yields:

\begin{tcbraster}[raster columns=2,raster equal height,nobeforeafter,raster column skip=2cm]
	\begin{library}{$L^{\Sigma}_{\text{CPA\$-Real}}$}
		\begin{lstlisting}[mathescape=true,autogobble=true]
			CTXT($m$):
			  if $\abs{m_l} \neq \abs{m_r}$:
			    error
			  return $\Sigma.enc(k, m)$
		\end{lstlisting}
	\end{library}
	\begin{library}{$\Sigma'$}
		\begin{lstlisting}[mathescape=true,autogobble=true]
			Enc(k, m):
			  $r \leftarrow \{0, 1\}^\lambda$
			  $s := r \oplus m$
			  $x \leftarrow \{0, 1\}^\lambda$
			  return (s, x)
		\end{lstlisting}
	\end{library}
\end{tcbraster}

Finally using the fact that the XOR of two binary strings one of which is
chosen from a uniform distribution, follows a uniform distribution once more,
and inlining the resulting code:

\begin{library}{$L^{\Sigma}_{\text{CPA\$-Rand}}$}
	\begin{lstlisting}[mathescape=true,autogobble=true]
		CTXT($m$):
		  if $\abs{m_l} \neq \abs{m_r}$:
		    error
		  $s \leftarrow \{0, 1\}^\lambda$
		  $x \leftarrow \{0, 1\}^\lambda$
		  return (s, x)
	\end{lstlisting}
\end{library}

This library finally returns two randoml values chosen from a uniform
distribution.  As such the encryption scheme possesses CPA\$ security.


\subsection{c)}

\subsubsection{Decryption algorithm}

\begin{library}{$\Sigma$}
	\begin{lstlisting}[mathescape=true,autogobble=true]
		Dec(k, (x, y)):
		  $s_1 = F^{-1}(k, x)$
		  $s_2 = F^{-1}(k, y)$
		  $m = s_2 \oplus s_1$
		  return m
	\end{lstlisting}
\end{library}

\subsubsection{CPA security}

Consider the following distinguisher:

\begin{library}{$B$}
	\begin{lstlisting}[mathescape=true,autogobble=true]
		$m_l = 0^\lambda$
		$m_r = 1^\lambda$

		$(x, y) := EAVESDROP(m_l, m_r)$

		return x == y
	\end{lstlisting}
\end{library}

By nature of its construction:
\begin{align*}
	P[B \diamond L^{\Sigma}_{\text{CPA-Left}} \rightarrow 1] & = P[F(k, s_1) = F(k, s_2)] \\
	& = P[F(k, s_1) = F(k, s_1 \oplus m)] \\
	& = P[s_1 = s_1 \oplus m] \tag*{As F is deterministic} \\
	& = 1\tag*{As $x \oplus 0^\lambda = x$ for all $\lambda$-bit strings $x$}
\end{align*}

And:
\begin{align*}
	P[B \diamond L^{\Sigma}_{\text{CPA-Right}} \rightarrow 1] & = P[F(k, s_1) = F(k, s_2)] \\
	& = P[s_1 = s_1 \oplus m] \tag*{As F is deterministic} \\
	& = 0 \tag*{As $x \oplus 1^\lambda \neq x$ for all $\lambda$-bit strings $x$}
\end{align*}

Hence $\lim_{\lambda \to \infty} Bias(B) = 1$ which is not negligible,
so the encryption scheme does not possess CPA-security.

\section{Modes of operation}

Let $m_i$ be the $\lambda$-bit blocks of plaintext, $c_i$ the corresponding
blocks of ciphertext.

\subsection{CBC}

As $c_i := F_k(m_i \oplus c_{i-1})$, a change in one input bit of $m_i$ will
affect all ciphertexts $c_j$ with $j \geq i$, that is all following blocks of
ciphertext.

Further, due to the properties of $F$, multiple `random' (uncorrelated with
which bit of the input was changed) bits of each ciphertext block will be
changed.

\subsection{OFB}

As each block of ciphertext is the XOR of the plaintext and the seed fed
i-times through $F$, changing one bit of a plaintext block $m_i$ will cause a
change in only the corresponding bit of the ciphertext block $c_i$.

\subsection{CTR}

As in the case of OFB, $m_i$ is only used in the calculation of $c_i$. Since
$c_i = m_i \oplus F_k(\$ + i - 1)$, changing one bit of $m_i$ will cause a
change in only the corresponding bit of $c_i$.

\end{document}


