\documentclass[a4paper]{scrreprt}

% Uncomment to optimize for double-sided printing.
% \KOMAoptions{twoside}

% Set binding correction manually, if known.
% \KOMAoptions{BCOR=2cm}

% Localization options
\usepackage[english]{babel}
\usepackage[T1]{fontenc}
\usepackage[utf8]{inputenc}

% Quotations
\usepackage{dirtytalk}

% Enhanced verbatim sections. We're mainly interested in
% \verbatiminput though.
\usepackage{verbatim}

% PDF-compatible landscape mode.
% Makes PDF viewers show the page rotated by 90°.
\usepackage{pdflscape}

% Advanced tables
\usepackage{tabu}
\usepackage{longtable}

% Fancy tablerules
\usepackage{booktabs}

% Graphics
\usepackage{graphicx}

% Current time
\usepackage[useregional=numeric]{datetime2}

% Float barriers.
% Automatically add a FloatBarrier to each \section
\usepackage[section]{placeins}

% Custom header and footer
\usepackage{fancyhdr}

\usepackage{geometry}
\usepackage{layout}

% Math tools
\usepackage{mathtools}
% Math symbols
\usepackage{amsmath,amsfonts,amssymb}
\usepackage{amsthm}

\DeclarePairedDelimiter\abs{\lvert}{\rvert}

\pagestyle{plain}
% \fancyhf{}
% \lhead{}
% \lfoot{}
% \rfoot{}
% 
% Source code & highlighting
\usepackage{listings}

% SI units
\usepackage[binary-units=true]{siunitx}
\DeclareSIUnit\cycles{cycles}

% Convenience commands
\newcommand{\mailsubject}{41100 - Cryptography - Series 4}
\newcommand{\maillink}[1]{\href{mailto:#1?subject=\mailsubject}
                               {#1}}

% Should use this command wherever the print date is mentioned.
\newcommand{\printdate}{\today}

\subject{41100 - Cryptography}
\title{Series 4}

\author{Michael Senn \maillink{michael.senn@students.unibe.ch} - 16-126-880}

\date{\printdate}

% Needs to be the last command in the preamble, for one reason or
% another. 
\usepackage{hyperref}


\begin{document}
\maketitle


\setcounter{chapter}{3}
\chapter{Series 4}

\section{Deterministic libraries}

Let $L_1$, $L_2$ be two deterministic libraries. Let $OUTPUT(f, x_1, \cdots,
x_n) := return f(x_1, \cdots, x_n)$ be a program which calls functions $f$ with
parameters $x := (x_1 \cdots x_n)$ and returns the result, when linked to
either $L_1$ or $L_2$.

As the two libraries are deterministic $OUTPUT$ is deterministic as well,
since its value is given by the choice of function $f$, and parameters $x$.

As the libraries are finite and deterministic, and an adversary has access to
the code of a library per Kerckhoff's principle, they can be analyzed in some
finite amount of time. An adversary can learn if there is any function $f$
which, when called with input $x_f$, produces two different outputs $y_1$ and
$y_2$.

If, for all functions $f$ and parameters $x_f$ the two libraries return the
same result, they are interchangable as no distinguishing algorithm is able to
differentiate the two.

If however there is a function $f$ and a set of parameters $x_f$ such that $y_1
:= OUTPUT(f, x) \diamond L_1) \neq OUTPUT(f, x) \diamond L_2) =: y_2$ then the
adversary can use these to clearly differentiate between the two libraries:

\begin{lstlisting}[mathescape=true, frame=single]
	f = ...
	$x_f$ = ...
	$y_1$ = ...
	$y_2$ = ...

	Differentiate():
	  y = f(x)
	  if y == $y_1$:
	    return 1
	  elsif y == $y_2$:
	    return 0
\end{lstlisting}

\section{Hash-function collisions}

Let $H: \{0, 1\}^{*} \Rightarrow \{0, 1\}^\lambda$ be an ideal hash function.

\subsection{$\lambda = 40$}

As the output of the hash function is uniformly distributed, the probability of
a collision if computing $10^6$ hashes can be modeled with the generalized
birthday problem:

\[
	\text{BirthdayProb}(10^6, 2^{40}) = 1 - \prod_{i = 1}^{10^6 - 1} \left(1 - \frac{1}{2^{40}}\right)
\]

As $10^6 \leq \sqrt(2) \cdot 2^{20} = \sqrt{2^{41}}$, we can calculate
asymptotic bounds using the method mentioned in the book. As such, with $q :=
10^6$ and $N := 2^{40}$:

\begin{align*}
	\SI{28.7}{\percent} & \approx 0.632 \cdot \frac{10^6 \cdot (10^6 - 1)}{2 \cdot 2^{40}} \\
			    & = 0.632 \cdot \frac{q (q-1)}{2N}  \\
			    & \leq BirthdayProb(q, N) \\ 
			    & \leq \frac{10^6 \cdot (10^6 - 1)}{2 \cdot 2^{40}} \\
			    & = \frac{q (q-1)}{2N}  \\
			    & \approx \SI{45.5}{\percent}
\end{align*}

In addition to the lower bound of \SI{28.7}{\percent} and the upper bound of
\SI{45.5}{\percent} we can calculate an approximate value using the first two
terms of the taylor series of $\exp(x)$, as shown in the lecture:

\[
	\text{BirthdayProb}(q, N) \approx 1 - \exp(-\frac{q \cdot (q - 1)}{2
	N}) \approx 1 - \exp(-\frac{q^2}{2 N}) \approx \SI{36.5}{\percent}
\]

\subsection{$\lambda = 256$}

We solve the bounds above for $q$ using the approximation that $q \cdot (q-1)
\approx q^2$ for big values of $q$. After discarding negative solutions this
gives approximative bounds for $q$ as:


\[
	\sqrt{2 N p} \leq q \leq \sqrt{\frac{250}{79} \cdot N p}
\]

Similarly, for the approximate value:
\begin{align*}
	& p \approx \exp(- \frac{q^2}{2N}) \\
	\Rightarrow & \exp(- \frac{q^2}{2N}) \approx 1 - p \\
	\Rightarrow & -\frac{q^2}{2N} \approx \ln(1 - p) \\
	\Rightarrow & q^2 \approx -\ln(1 - p) \cdot 2N \\
	\Rightarrow & q \approx \sqrt{-\ln(1 - p) \cdot 2N}
\end{align*}

For $N := 2^{256}$, $p := 0.5$ this results in bounds of $3.40 * 10^{38} \leq q
\leq 4.28 * 10^{38}$, and an approximate value of $q \approx 4.01 * 10^{38}$.


\section{Salt}

Assuming that the used hashed function does not produce any collisions, the
password database containing two equal hashes happens exactly if two users
chose the same password. This can once more be understood as an instance of the
birthday problem.

\subsection{Password collisions}

Let $p := 0.5$ and $N := 2^{20}$. Using the same approximations as above
results in approximative bounds of $1024 \leq q \leq 1288$, and an approximate
value of $q \approx 1206$.

\subsection{Salting passwords}

As the salt is appended to the password before hashing, passwords are uniformly
distributed over $\{0, 1\}^{20}$ and salts are uniformly distributed over $\{0,
1\}^{256}$, the resulting concatenation before hashing is uniformly distributed
over $\{0, 1\}^{276}$.

Using the same approximations as above results in approximative bounds of $3.48
* 10^{41} \leq q \leq 4.38 * 10^{41}$, and an approximate value of $q \approx
4.10 * 10^{41}$

\end{document}

