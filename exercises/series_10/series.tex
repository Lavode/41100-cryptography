\documentclass[a4paper]{scrreprt}

% Uncomment to optimize for double-sided printing.
% \KOMAoptions{twoside}

% Set binding correction manually, if known.
% \KOMAoptions{BCOR=2cm}

% Localization options
\usepackage[english]{babel}
\usepackage[T1]{fontenc}
\usepackage[utf8]{inputenc}

% Quotations
\usepackage{dirtytalk}

% Enhanced verbatim sections. We're mainly interested in
% \verbatiminput though.
\usepackage{verbatim}

% Automatically remove leading whitespace in lstlisting
\usepackage{lstautogobble}

% PDF-compatible landscape mode.
% Makes PDF viewers show the page rotated by 90°.
\usepackage{pdflscape}

% Advanced tables
\usepackage{tabu}
\usepackage{longtable}

% Fancy tablerules
\usepackage{booktabs}

% Graphics
\usepackage{graphicx}

% Current time
\usepackage[useregional=numeric]{datetime2}

% Float barriers.
% Automatically add a FloatBarrier to each \section
\usepackage[section]{placeins}

% Custom header and footer
\usepackage{fancyhdr}

\usepackage{geometry}
\usepackage{layout}

% Math tools
\usepackage{mathtools}
% Math symbols
\usepackage{amsmath,amsfonts,amssymb}
\usepackage{amsthm}
% General symbols
\usepackage{stmaryrd}

\DeclarePairedDelimiter\abs{\lvert}{\rvert}

% Indistinguishable operator (three stacked tildes)
\newcommand*{\diffeo}{% 
  \mathrel{\vcenter{\offinterlineskip
  \hbox{$\sim$}\vskip-.35ex\hbox{$\sim$}\vskip-.35ex\hbox{$\sim$}}}}

% Bullet point
\newcommand{\tabitem}{~~\llap{\textbullet}~~}

\pagestyle{plain}
% \fancyhf{}
% \lhead{}
% \lfoot{}
% \rfoot{}
% 
% Source code & highlighting
\usepackage{listings}

% Coloured boxes!
\usepackage[most]{tcolorbox}
\newtcolorbox{library}[2][]{
	enhanced,
	sharp corners,
	coltitle=black, % title colour
	colbacktitle=black!10!white, % title bg colour
	halign title=center, % title align
	toptitle=1mm, % Top/bottom additional spacing for title
	bottomtitle=1mm,
	fonttitle=\ttfamily,
	colback=white, % body bg colour
	fontupper=\ttfamily,
	title=#2,#1
}

\newtcolorbox{boxcomment}[2][]{
	enhanced,
	colframe=white, % frame colour
	colbacktitle=white, % title bg colour
	halign=center, % body align
	colback=white, % body bg colour
	fonttitle=\ttfamily,
	fontupper=\ttfamily,
	title=#2,#1
}

% SI units
\usepackage[binary-units=true]{siunitx}
\DeclareSIUnit\cycles{cycles}

% Convenience commands
\newcommand{\mailsubject}{41100 - Cryptography - Series 10}
\newcommand{\maillink}[1]{\href{mailto:#1?subject=\mailsubject}
                               {#1}}

% Should use this command wherever the print date is mentioned.
\newcommand{\printdate}{\today}

\subject{41100 - Cryptography}
\title{Series 10}

\author{Michael Senn \maillink{michael.senn@students.unibe.ch} - 16-126-880}

\date{\printdate}

% Needs to be the last command in the preamble, for one reason or
% another. 
\usepackage{hyperref}


\begin{document}
\maketitle


\setcounter{chapter}{9}
\chapter{Series 10}

\section{ElGamal encryption}


\subsection{Product of messages}

Given $c_1 := (g^{r_1}, m_1 \cdot Y^{r_1})$, $c_2 := (g^{r_2}, m_2 \cdot
Y^{r_2})$. 

Let $R := g^{r_1} \cdot g^{r_2} = g^{r_1 + r_2}$, and $C := m_1 Y^{r_1} \cdot
m_2 Y^{r_2} = m_1 m_2 \cdot Y^{r_1 + r_2}$. Then:
\begin{align*}
  Dec(x, (R, C)) & = C / R^x \\
                 & = \frac{m_1 m_2 \cdot Y^{r_1 + r_2}}{{g^{r_1 + r_2}}^{x}} \\
                 & = \frac{m_1 m_2 \cdot g^{x (r_1 + r_2)}}{g^{(r_1 + r_2)x}} \\
                 & = m_1 \cdot m_2
\end{align*}

\subsection{Plaintext collision}

Given $Y, x := KeyGen()$, $R_1, C_1 := Enc(Y, m)$, pick $r_2 \leftarrow
\mathbb{Z}_q$ such that $Y^{r_2} \neq Y^{r_1}$. Let $R_2, C_2 := (g^{r_2}, m \cdot
Y^{r_2})$.

Then:
\[
  C_2 = m \cdot Y^{r_2} \neq m \cdot Y^{r_1} = C_1
\]

And:
\[
  Dec(x, (R_2, C_2)) = m = Dec(x, (R_1, C_1))
\]

\subsection{Reusing $R$}

If $R = g^r$ is reused to encrypt two messages $m_1, m_2$, then $C_1 = m_1
\cdot Y^r$, and $C_2 = m_2 \cdot Y^r$. As such:

\[
  \frac{C_1}{C_2} = \frac{m_1 \cdot Y^r}{m_2 \cdot Y^r} = \frac{m_1}{m_2}
\]

This allows to derive one message based on its cipher text if the other message
is known.

\section{Ciphertext size of CPA-secure public-key encryption}

If the length of the ciphertext is $\alpha \log(\lambda)$, then there exist at
most $n := 2^{\alpha \log(\lambda)}$ cipher texts. Note that for sufficiently large
values of $\lambda$, $2^{\log(\lambda)} \leq \lambda$ - with equality in the
case of the logarithm with base $2$ -  as such $n \leq
\lambda^\alpha$ is bounded by a polynomial function.

If the number of ciphertexts is bounded by a polynomial function, a polynomial
attacker is able to enumerate a non-negligible quantity of the cipher text
space. Consider the following distinguisher:

\begin{library}{$A$}
	\begin{lstlisting}[mathescape=true,autogobble=true]
		# Expected value of coupon collector's problem
		lookup_table_size := $n \cdot H_n$
		known_ctxts := () # Empty set

		$p_k := GETKEY()$

		for i = 1 to lookup_table_size:
		  known_ctxts << $\Sigma$.Enc($p_k$, 0)

		c := TEST(0, 1)

		if c in known_ctxts:
		  return 1
		else:
		  return 0
	\end{lstlisting}
\end{library}

If $n$ is bounded by a polynomial function then the expected value of the
coupon collector's problem $n \cdot H_n$, with $H_n$ the n-th harmonic number,
is bounded by a polynomial function too, so such an enumeration can be done.

As such:

\[
	P\left[A \diamond L^\Sigma_{\text{pk-CPA-R}} \rightarrow 1\right] = 0
\]

And with the chosen value of $n$:

\[
	P\left[A \diamond L^\Sigma_{\text{pk-CPA-L}} \rightarrow 1\right] = 0.5
\]

This distinguisher has an advantage of $Bias(A) = 0.5$, so $\Sigma$ does not
possess pk-CPA security.

\section{Unbounded power}

Consider the following brute-force algorithm which, given a ciphertext $c$ and
public key $p_k$ computes the corresponding plaintext $m$:

\begin{library}{$A$}
	\begin{lstlisting}[mathescape=true,autogobble=true]
		COMPUTE($c, p_k$):
		  while True:
		    if $c = \Sigma.Enc(p_k, 0)$:
		      return 0
		    if $c = \Sigma.Enc(p_k, 1)$:
		      return 1
	\end{lstlisting}
\end{library}

Such an algorithm clearly has non-polynomial running time as the size of the
space of cipher texts is not polynomial in the security parameter $\lambda$.
It is, however guaranteed to terminate in finite time, as each attempt at
computing the plaintext has a probability of success greater than zero, so the
cumulative probability of success approaches $1$ as the amount of iterations
increases. In other words the probability of not terminating approaches zero as
the number of iterations approaches infinity.

Other possibly more efficient alternatives exist as well, such as repeatedly  
calling $(p'_k, s'_k) := KeyGen()$ until $p'_k = p_k$, in which case $s'_k$ is
the used secret key.

\end{document}
