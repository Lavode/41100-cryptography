\documentclass[a4paper]{scrreprt}

% Uncomment to optimize for double-sided printing.
% \KOMAoptions{twoside}

% Set binding correction manually, if known.
% \KOMAoptions{BCOR=2cm}

% Localization options
\usepackage[english]{babel}
\usepackage[T1]{fontenc}
\usepackage[utf8]{inputenc}

% Quotations
\usepackage{dirtytalk}

% Enhanced verbatim sections. We're mainly interested in
% \verbatiminput though.
\usepackage{verbatim}

% PDF-compatible landscape mode.
% Makes PDF viewers show the page rotated by 90°.
\usepackage{pdflscape}

% Advanced tables
\usepackage{tabu}
\usepackage{longtable}

% Fancy tablerules
\usepackage{booktabs}

% Graphics
\usepackage{graphicx}

% Current time
\usepackage[useregional=numeric]{datetime2}

% Float barriers.
% Automatically add a FloatBarrier to each \section
\usepackage[section]{placeins}

% Custom header and footer
\usepackage{fancyhdr}

\usepackage{geometry}
\usepackage{layout}

% Math tools
\usepackage{mathtools}
% Math symbols
\usepackage{amsmath,amsfonts,amssymb}
\usepackage{amsthm}

\DeclarePairedDelimiter\abs{\lvert}{\rvert}

\pagestyle{plain}
% \fancyhf{}
% \lhead{}
% \lfoot{}
% \rfoot{}
% 
% Source code & highlighting
\usepackage{listings}

% Convenience commands
\newcommand{\mailsubject}{41100 - Cryptography - Series 2}
\newcommand{\maillink}[1]{\href{mailto:#1?subject=\mailsubject}
                               {#1}}

% Should use this command wherever the print date is mentioned.
\newcommand{\printdate}{\today}

\subject{41100 - Cryptography}
\title{Series 2}

\author{Michael Senn \maillink{michael.senn@students.unibe.ch} - 16-126-880}

\date{\printdate}

% Needs to be the last command in the preamble, for one reason or
% another. 
\usepackage{hyperref}


\begin{document}
\maketitle


\setcounter{chapter}{1}
\chapter{Series 2}

\section{Basics on libraries}

\subsection{Programs and libraries}

Note that the output of $L_1$ is distributed by $X \sim \{0, 1, \cdots, n-1\}$
Let $X, Y$ iid as per above. Then:

\begin{itemize}
	\item $P[A_1 \diamond L_1 \Rightarrow 1] = P[X = Y] = \frac{1}{6}$
	\item $P[A_1 \diamond L_2 \Rightarrow 1] = P[0 = 0] = 1$
	\item $P[A_2 \diamond L_1 \Rightarrow 1] = P[X \geq 3] = \frac{3}{6}$
	\item $P[A_2 \diamond L_2 \Rightarrow 1] = P[0 \geq 3] = 0$

\end{itemize}

\subsection{Exchangability of libraries}

\subsubsection{Bitwise complement}

Note first that $L_{\text{left}}$ produces a uniform distribution on $\{0,
1\}^n$. As the bitwise complement is a bijection of $\{0, 1\}^n$ on itself, it
follows that $L_{\text{right}}$ is also uniformly distributed on that same set.
Hence $L_{\text{left}} \equiv L_{\text{right}}$ independent of the choice of
$n$.

\subsubsection{2x \% n}

Note that, for $n$ even, $L_{\text{right}}$ is uniformly distributed on:

\[
	2 \cdot Z_n \mod n = \{0, 2, \cdots, n - 2, n, n + 2, \cdots, 2n - 2\} \mod n = \{0, 2, \cdots, n - 2, 0, 2, \cdots, n - 2\}.
\]

For $n$ odd, $L_{\text{right}}$ is uniformly distributed on:

\[
	2 \cdot Z_n \mod n = \{0, 2, \cdots, n - 1, n + 1, \cdots, 2n - 2\} \mod n = \{0, 2, \cdots, n - 1, 1, 3, \cdots, n - 2\}.
\]

Clearly, for $n$ odd, the output follows the same distribution as
$L_{\text{left}}$, so $L_{\text{left}} \equiv L_{\text{right}}$. For $n$ even,
however, $L_{\text{right}}$ is uniformly distributed on $\{0, 2, \cdots,
n-2\}$.

A distinguisher $A$ can return $1$ if the given value is even, $0$ if it is
odd. Then, $P[A \diamond L_{\text{left}} \Rightarrow 1] = \frac{1}{2} \neq P[A
\diamond L_{\text{right}} \Rightarrow 1] = 1$. Hence $L_{\text{left}}$ and
$L_{\text{right}}$ are not exchangable.

\subsubsection{x AND y}

As before, $L_{\text{right}}$ is uniformly distributed on $\{0, 1\}^n$. By the
definition of the bitwise AND operation, for any bit $x_i$ in the output,
$P[x_i = 0] = \frac{3}{4}$, and $P[x_i = 1] = \frac{1}{4}$.

Let $k$ be the number of bits equal to $1$. In the case of $L_{\text{left}}$, k
follows a binomial distribution $Binom(n, \frac{1}{4})$, in the case of
$L_{\text{right}}$ it follows a binomial distribution $Binom(n, \frac{1}{2})$.

This allows solving for the value of $k$ at which the likelihood of the output
$x$ being produced by $L_{\text{left}}$ is equal to it being produced by
$L_{\text{right}}$:

\begin{align*}
	& \binom{n}{k'} \left(\frac{1}{4}\right)^{k'} \left(\frac{3}{4}\right)^{n - k'} = \binom{n}{k'} \left(\frac{1}{2}\right)^n \\
	\Rightarrow & \left(\frac{1}{4}\right)^{k'} \cdot \left(\frac{3}{4}\right)^{n - k'} = \left(\frac{1}{2}\right)^n \\
	\Rightarrow & k' = \frac{n \cdot \log{\frac{3}{2}}}{\log{3}}
\end{align*}

A distinguisher $A$ can then output $1$ if $k > k'$, and $0$ if $k \leq k'$. It
then follows that $P[A \diamond L_{\text{left}} \Rightarrow 1] \leq P[A
\diamond L_{\text{right}} \Rightarrow 1]$, with equality only for special
values of $n$ and $k$.

\end{document}

