\documentclass[a4paper]{scrreprt}

% Uncomment to optimize for double-sided printing.
% \KOMAoptions{twoside}

% Set binding correction manually, if known.
% \KOMAoptions{BCOR=2cm}

% Localization options
\usepackage[english]{babel}
\usepackage[T1]{fontenc}
\usepackage[utf8]{inputenc}

% Quotations
\usepackage{dirtytalk}

% Enhanced verbatim sections. We're mainly interested in
% \verbatiminput though.
\usepackage{verbatim}

% PDF-compatible landscape mode.
% Makes PDF viewers show the page rotated by 90°.
\usepackage{pdflscape}

% Advanced tables
\usepackage{tabu}
\usepackage{longtable}

% Fancy tablerules
\usepackage{booktabs}

% Graphics
\usepackage{graphicx}

% Current time
\usepackage[useregional=numeric]{datetime2}

% Float barriers.
% Automatically add a FloatBarrier to each \section
\usepackage[section]{placeins}

% Custom header and footer
\usepackage{fancyhdr}

\usepackage{geometry}
\usepackage{layout}

% Math tools
\usepackage{mathtools}
% Math symbols
\usepackage{amsmath,amsfonts,amssymb}
\usepackage{amsthm}

\DeclarePairedDelimiter\abs{\lvert}{\rvert}

\pagestyle{plain}
% \fancyhf{}
% \lhead{}
% \lfoot{}
% \rfoot{}
% 
% Source code & highlighting
\usepackage{listings}

% SI units
\usepackage[binary-units=true]{siunitx}
\DeclareSIUnit\cycles{cycles}

% Convenience commands
\newcommand{\mailsubject}{41100 - Cryptography - Series 3}
\newcommand{\maillink}[1]{\href{mailto:#1?subject=\mailsubject}
                               {#1}}

% Should use this command wherever the print date is mentioned.
\newcommand{\printdate}{\today}

\subject{41100 - Cryptography}
\title{Series 3}

\author{Michael Senn \maillink{michael.senn@students.unibe.ch} - 16-126-880}

\date{\printdate}

% Needs to be the last command in the preamble, for one reason or
% another. 
\usepackage{hyperref}


\begin{document}
\maketitle


\setcounter{chapter}{2}
\chapter{Series 3}

\section{Nested encryption scheme}

We will show the one-time secrecy of $\Sigma^2$ by showing that
$L^{\Sigma^2}_{\text{OTS}_L} \equiv L^{\Sigma^2}_{\text{OTS}_L}$.

Consider the following library, as used in the definition of OTS:

\begin{lstlisting}[mathescape=true, frame=single]
	$L^{\Sigma^2}_{\text{OTS}_L}$:
	  Eavesdrop($m_l$, $m_r$):
	    k <- $\Sigma^2$.KeyGen()
	    c <- $\Sigma^2$.enc(k, m)
	    return c
\end{lstlisting}	

We start by factoring out certain statements into a separate library, the
composition of which is clearly equivalent to the above:

\begin{lstlisting}[mathescape=true, frame=single]
	$L_{\text{hyb}_1}$:
	  Eavesdrop($m_l$, $m_r$):
	    c := CTXT($m_l$)
	    return c

	$\diamond$

	$L^{\Sigma^2}_{\text{OTS\$}-\text{real}}$:
	  CTXT(m):
	    k <- $\Sigma^2$.KeyGen()
	    c <- $\Sigma^2$.enc(k, m)
	    return c
\end{lstlisting}	

We then inline the two $\Sigma^2$ functions per their definition:

\begin{lstlisting}[mathescape=true, frame=single]
	$L_{\text{hyb}_2}$:
	  Eavesdrop($m_l$, $m_r$):
	    c := CTXT($m_l$)
	    return c

	$\diamond$

	$L^{\Sigma^2}_{\text{OTS\$}-\text{real}'}$:
	  CTXT(m):
	    $k_1$ <- $\Sigma$.KeyGen()
	    $k_2$ <- $\Sigma$.KeyGen()

	    $c_1$ <- $\Sigma$.enc(k_1, m)
	    $c_2$ <- $\Sigma$.enc(k_2, $c_1$)
	    return $c_2$
\end{lstlisting}	

Rearranging the lines of $CTXT(m)$ allows to extract two subsequent invocations
of $CTXT'(m)$ of the $L^\Sigma_{\text{OTS\textdollar-real}}$ library:

\begin{lstlisting}[mathescape=true, frame=single]
	$L_{\text{hyb}_3}$:
	  Eavesdrop($m_l$, $m_r$):
	    c := CTXT($m_l$)
	    return c

	$\diamond$

	$L^{\Sigma^2}_{\text{OTS\$}-\text{real}'}$:
	  CTXT(m):
	    $c_1$ <- CTXT'(m)
	    $c_2$ <- CTXT'($c_1$)
	    return $c_2$

	$\diamond$

	$L^\Sigma_{\text{OTS\$}-\text{real}}$:
	  CTXT'(m):
	    k <- $\Sigma$.KeyGen()
	    c <- $\Sigma$.enc(k, m)
	    return $c$
\end{lstlisting}	

As $\Sigma$ satisfies OTS, $L^\Sigma_{\text{OTS\$}-\text{real}} \equiv
L^\Sigma_{\text{OTS\$}-\text{rand}}$. We can then replace one library with the
other:

\begin{lstlisting}[mathescape=true, frame=single]
	$L_{\text{hyb}_4}$:
	  Eavesdrop($m_l$, $m_r$):
	    c := CTXT($m_l$)
	    return c

	$\diamond$

	$L^{\Sigma^2}_{\text{OTS\$}-\text{real}'}$:
	  CTXT(m):
	    $c_1$ <- CTXT'(m)
	    $c_2$ <- CTXT'($c_1$)
	    return $c_2$

	$\diamond$

	$L^\Sigma_{\text{OTS\$}-\text{rand}}$:
	  CTXT'(m):
	    c <- $\Sigma$.C
	    return $c$
\end{lstlisting}	

As the output does not depend on the value passed to $CTXT'$ and $CTXT$
anymore, we can replace $m_l$ with $m_r$ in the $Eavesdrop()$ function.
Afterwards all steps above can be undone in reverse order, finally showing that
$L^{\Sigma^2}_{\text{OTS}_L} \equiv L^{\Sigma^2}_{\text{OTS}_L}$. Hence
$\Sigma^2$ satisfies one-time secrecy.

\section{Negligible functions}

A function $f$ is negligible exactly if, for all polynomials $p(\lambda)$,
the following holds:

\[
	\lim_{\lambda \rightarrow \infty} p(\lambda) f(\lambda) = 0
\]

As per claim 4.3 in the book it is sufficient to show this for polynomials of
the form $\lambda^c, c \in \mathbb{Z}$.

\subsection{Examples}

\subsubsection{$2^{\frac{-\lambda}{2}}$}

Note that $2^{\frac{\lambda}{2}}$ is an exponential. As, for sufficiently large
values of $\lambda$, any exponential will grow faster than any polynomial it
follows that:

\[
	\lim_{\lambda \rightarrow \infty} p(\lambda) \frac{1}{2^{\frac{\lambda}{2}}} = 0
\]

Hence $f$ is negligible.

\subsubsection{$\lambda^{-2}$}

Let $p(\lambda) := \lambda^3$. Then:

\[
	\lim_{\lambda \rightarrow \infty} p(\lambda) \frac{1}{\lambda^2} = \lambda \neq 0
\]

Hence $f$ is not negligible.

\subsubsection{$\lambda^{-\frac{1}{\lambda}}$}

Note first that, using L'Hospital's rule, $\lim_{\lambda \rightarrow \infty}
\sqrt[\lambda]{\lambda} = 1$. Let $p(\lambda) := \lambda$, then:

\[
	\lim_{\lambda \rightarrow \infty} p(\lambda) \frac{1}{\sqrt[\lambda]{\lambda}} = 1 \neq 0
\]

Hence $f$ is not negligible.

\subsubsection{$\lambda^{-\frac{1}{2}}$}

Let $p(\lambda) := \sqrt{\lambda}$, then:

\[
	\lim_{\lambda \rightarrow \infty} p(\lambda) \frac{1}{\sqrt{\lambda}} = 1 \neq 0
\]

Hence $f$ is not negligible.

\subsubsection{$2^{-\sqrt{\lambda}}$}

As $\sqrt{\lambda}$ is not bound, $2^{\sqrt{\lambda}}$ is an exponential
function. Using the same reasoning as for the first example, $f$ is hence
negligible.

\subsection{Multiplication and division of negligible functions}

\subsubsection{Multiplication of negligible functions}

\begin{proof}
	Let $f(\lambda)$, $g(\lambda)$ be negligible. Let $p(\lambda)$ be any
	polynomial function. We aim to show that $f \cdot g$ is negligible.

	Note first that, if $f$ is negligible, the following holds for all
	constants $c \in \mathbb{R}$, in particular also for $c = 1$:
	\begin{align*}
		& \lim_{\lambda \rightarrow \infty} p(\lambda) f(\lambda) = 0 \\
		\Rightarrow & \lim_{\lambda \rightarrow \infty} c \cdot f(\lambda) = 0 \\
		\Rightarrow & \lim_{\lambda \rightarrow \infty} f(\lambda) = 0
	\end{align*}

	As such:
	\begin{align*}
		\lim_{\lambda \rightarrow \infty} p(\lambda) \cdot (f(\lambda) \cdot g(\lambda)) 
		  & = \lim_{\lambda \rightarrow \infty} (p(\lambda) \cdot f(\lambda)) \cdot g(\lambda) & \text{Associativity of multiplication} \\
		  & = \lim_{\lambda \rightarrow \infty} (p(\lambda) \cdot f(\lambda)) \cdot 0 & \text{Negligibility of $g$} \\
		  & = \lim_{\lambda \rightarrow \infty} 0 \cdot 0 & \text{Negligibility of $f$} \\
		  & = 0
	\end{align*}

	Hence $f \cdot g$ is negligible.
\end{proof}

\subsubsection{Division of negligible functions}

Let $f(\lambda) := \frac{1}{2^\lambda}$, $g(\lambda) := \frac{1}{4^\lambda}$.
Clearly both $f$ and $g$ are negligible, as seen in the previous exercise.
However:

\[
	\frac{f(\lambda)}{g(\lambda)} = \frac{\frac{1}{2^\lambda}}{\frac{1}{4^\lambda}} = 2^\lambda
\]

which, by nature of being an exponential function, is not negligible.

\section{Hashrate}

\subsection{AVX1 hashrate}

Based on the linked document we can estimate the performance, in cycles per
byte, of a single-threaded AVX1 CPU when processing \SI{1}{\mega\byte} blocks to be on
the order of \SI{12.8}{\cycles \per \byte}.

We then get the amount of cycles for a \SI{1}{\mega\byte} block as
$\SI{12.8}{\cycles \per \byte} \cdot \SI{1}{\mega\byte} = \SI{1.2e7}{\cycles}$,
and the hash rate as $\SI{e9}{\cycles \per \second} / \SI{1.2e7}{\cycles} =
\SI{167}{hashes \per \second}$.

\subsection{Comparison with Bitcoin hashrate}

The current seven-day average hashrate of the Bitcoin network is around
\SI{9.1e19}{hashes \per \second}. To achieve such a hash rate with AVX1 CPUs
one would need $\SI{9.1e19}{hashes \per \second} / \SI{167}{hashes \per
\second} \approx \SI{5.4e17}{CPUs}$

\section{A random cipher}

\subsection{Formalizing $\Sigma$}

 Let $n$ be the length of keys $\Sigma$ uses. Its algorithms can then be
 formalized as:

\begin{lstlisting}[mathescape=true, frame=single]
	$\Sigma.M := \{0, 1\}^k$
	$\Sigma.C := \{0, 1\}^k$
	$\Sigma.K := \{0, 1\}^n$

	$\Sigma$.KeyGen():
	  $k \leftarrow \Sigma.K$
	  return k

	$\Sigma.Enc$($k \in \Sigma.K$, $m \in \Sigma.M$):
	  c := DO_SOMETHING_WITH_K_AND_M()
	  return c

	$\Sigma.Dec$($k \in \Sigma.K$, $c \in \Sigma.C$):
	  m := DO_SOMETHING_WITH_K_AND_C()
	  return m
	
\end{lstlisting}


\subsection{Brute-forcing a message}

If an adversary wants to brute-force a plaintext message $m \leftarrow \{0,
1\}^k$, given $c := ENCRYPT(m)$, he can attempt doing so by picking random
messages $m' \leftarrow \{0, 1\}^k$, calling $c' := ENCRYPT(m')$, and checking
if $c' = c$. If so, $m' = m$.

Let $n := 2^k$. For a single $m' <- \{0, 1\}^k$, the chance of $m \neq m'$ is
given as  $P(m' \neq m) = \frac{n-1}{n}$. Asuming the adversary takes care not
to try the same plaintext message multiple times, the probability of failure on
the second attempt is $\frac{n - 2}{n - 1}$, and so on. For $q \leq n$ attempts
this results in a probability of success as:

\begin{align*}
	1 - \frac{(n - 1) \cdot (n - 2) \cdots (n - q)}{n \cdot (n-1) \cdots (n - q + 1)} 
	  & = 1 - \frac{(n-1)! \cdot (n - q)!}{(n - q - 1)! \cdot n!} \\
	  & = 1 - \frac{n - q}{n} \\
	  & = 1 - 1 - \frac{q}{n} \\
	  & = \frac{q}{n}
\end{align*}

Were the adversary to pick at random without excluding duplicates, the
probability of success would instead be $1 - (1 - \frac{1}{n})^q <
\frac{q}{n}$.

Hence, $\frac{q}{n}$ is an upper limit on the probability of the adversary
guessing $m$ in $q$ attempts, if given $c$ and access to $ENCRYPT(m \in M)$.
\end{document}

