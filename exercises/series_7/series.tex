\documentclass[a4paper]{scrreprt}

% Uncomment to optimize for double-sided printing.
% \KOMAoptions{twoside}

% Set binding correction manually, if known.
% \KOMAoptions{BCOR=2cm}

% Localization options
\usepackage[english]{babel}
\usepackage[T1]{fontenc}
\usepackage[utf8]{inputenc}

% Quotations
\usepackage{dirtytalk}

% Enhanced verbatim sections. We're mainly interested in
% \verbatiminput though.
\usepackage{verbatim}

% PDF-compatible landscape mode.
% Makes PDF viewers show the page rotated by 90°.
\usepackage{pdflscape}

% Advanced tables
\usepackage{tabu}
\usepackage{longtable}

% Fancy tablerules
\usepackage{booktabs}

% Graphics
\usepackage{graphicx}

% Current time
\usepackage[useregional=numeric]{datetime2}

% Float barriers.
% Automatically add a FloatBarrier to each \section
\usepackage[section]{placeins}

% Custom header and footer
\usepackage{fancyhdr}

\usepackage{geometry}
\usepackage{layout}

% Math tools
\usepackage{mathtools}
% Math symbols
\usepackage{amsmath,amsfonts,amssymb}
\usepackage{amsthm}

\DeclarePairedDelimiter\abs{\lvert}{\rvert}

% Indistinguishable operator (three stacked tildes)
\newcommand*{\diffeo}{% 
  \mathrel{\vcenter{\offinterlineskip
  \hbox{$\sim$}\vskip-.35ex\hbox{$\sim$}\vskip-.35ex\hbox{$\sim$}}}}

\pagestyle{plain}
% \fancyhf{}
% \lhead{}
% \lfoot{}
% \rfoot{}
% 
% Source code & highlighting
\usepackage{listings}

% Coloured boxes!
\usepackage[most]{tcolorbox}
\newtcolorbox{library}[2][]{
	enhanced,
	sharp corners,
	coltitle=black, % title colour
	colbacktitle=black!10!white, % title bg colour
	halign title=center, % title align
	toptitle=1mm, % Top/bottom additional spacing for title
	bottomtitle=1mm,
	fonttitle=\ttfamily,
	colback=white, % body bg colour
	fontupper=\ttfamily,
	title=#2,#1
}

\newtcolorbox{boxcomment}[2][]{
	enhanced,
	colframe=white, % frame colour
	colbacktitle=white, % title bg colour
	halign=center, % body align
	colback=white, % body bg colour
	fonttitle=\ttfamily,
	fontupper=\ttfamily,
	title=#2,#1
}

% SI units
\usepackage[binary-units=true]{siunitx}
\DeclareSIUnit\cycles{cycles}

% Convenience commands
\newcommand{\mailsubject}{41100 - Cryptography - Series 7}
\newcommand{\maillink}[1]{\href{mailto:#1?subject=\mailsubject}
                               {#1}}

% Should use this command wherever the print date is mentioned.
\newcommand{\printdate}{\today}

\subject{41100 - Cryptography}
\title{Series 7}

\author{Michael Senn \maillink{michael.senn@students.unibe.ch} - 16-126-880}

\date{\printdate}

% Needs to be the last command in the preamble, for one reason or
% another. 
\usepackage{hyperref}


\begin{document}
\maketitle


\setcounter{chapter}{6}
\chapter{Series 7}

Let $\Sigma := \{0, 1\}$.

\section{PRF using PRG}

Let $F: \Sigma^\lambda \times \Sigma^{2 \lambda} \rightarrow \Sigma^{2
\lambda}$ be a secure PRF, $G: \Sigma^\lambda \rightarrow \Sigma^{2 \lambda}$ a
secure PRG, $F': \Sigma^\lambda \times \Sigma^\lambda \rightarrow \Sigma^{2
\lambda}: x \rightarrow F(k, G(x))$ a candidate PRG.

As $G(x) \in \Sigma^{2\lambda}$ for all input values $x \in \Sigma^\lambda$,
$G(x)$ is guaranteed to be a valid input for $F$. Further as $G$ is injective,
there exists for all $y \in \Sigma^{2 \lambda}$ at most one $x \in
\Sigma^\lambda$ such that $G(x) = y$.

With this in mind, choosing any $x$ in $\Sigma^\lambda$, and using $G(x)$ as
input for $F$ is no different from choosing any $x$ in the image of $G$ and
using it as input for $F$.

As $F$ is secure for all valid inputs, $F'(k, x) = F(k, G(x))$ is therefore
also secure for all choices of $x$, and is hence a secure PRF.

\subsection{Security of $F'$}


\subsection{Insecurity of $F'$ if constructed with $\tilde{G}$} 

Consider the following distinguisher:

\begin{library}{$B$}
	\begin{lstlisting}[mathescape=true]
		$x \leftarrow \{0, 1\}^{\lambda - 1}$
		$x_1 := x || 0$
		$x_2 := x || 1$

		return LOOKUP($x_1$) == LOOKUP($x_2$)
	\end{lstlisting}
\end{library}

By the nature of its construction:
\[
	P[B \diamond L^{F'}_{\text{PRG-real}} \rightarrow 1] = 1
\]

And:
\[
	P[B \diamond L^{F'}_{\text{PRG-rand}} \rightarrow 1] = 2^{-2\lambda}
\]

Hence:
\[
	\lim_{\lambda \to \infty} Bias(B) = \lim_{\lambda \to \infty} 1 - 2^{-2\lambda} = 1
\]

As $\lim_{\lambda \to \infty} Bias(B) \neq 0$ it is non-negligible, so F' is
not secure.

\section{Pseudo-random permutations}

Let $\alpha: \Sigma^\mu \rightarrow \Sigma^\mu$ be a random permutation on
$\Sigma^\mu$. Let $F$ be a secure PRP, $k \in \Sigma^\lambda$ its key.

We assume that `the probability that the chosen permutation agrees with some
permutation of F' refers to the existence of at least one element which is
mapped to the same element by both permutations - that is $\exists x \in
\Sigma^\mu : F(k, x) = \alpha(x)$.

As a permutation can be thought of as an ordererd arrangment of elements in its
set, the probability of two permutations having $n$ elements in the same
position is equal to the probability of one of the two permutations having $n$
fixed points.

For a set of $n$ elements there exist $n!$ permutations which is shown
trivially. The amount of permutations without fixed points - called
derangements - is denoted by $!n = (n - 1)(!(n - 1) + !(n - 2))$. According to
literature this is known to be equal to:
\[
	!n = n! \sum_{i = 0}^{n} \frac{(-1)^i}{i!}
\]


\subsection{Probability of conflict}

As such the possibility of a given permutation $\alpha$ on $\Sigma^\mu$ having
at least one fixed point is, with $n := 2^\mu$:
\begin{align*}
	P[\alpha \text{ has at least one fixed point}] & = 1 - \frac{!n}{n!} \\
	& = 1 - \frac{n! \sum_{i = 0}^{n} \frac{(-1)^i}{i!}}{n!} \\
	& = 1 - \sum_{i = 0}^{n} \frac{(-1)^i}{i!} \\
	& = 1 - \sum_{i = 0}^{2^\mu} \frac{(-1)^i}{i!}
\end{align*}

As $2^\mu$ grows expoenentially fast with $\mu$, and the absolute value of the
terms of the sum shrink faster still due to the presence of the factorial in
the divisor, the limit of the probability for $n \to \infty$ is of interest
too:

\begin{align*}
	\lim_{n \to \infty} 1 - \sum_{i = 0}^{n} \frac{(-1)^i}{i!} & = 1 - \lim_{n \to \infty} \sum_{i = 0}^{n} \frac{(-1)^i}{i!} \\
	& = 1 - \exp(-1) && \text{By the definition of the exponential function} \\
	& \approx 0.63
\end{align*}

\subsection{$\lambda = \mu = 128$}

Due to the speed at which the series converges to its limit - for $n = 10$ the
difference is already less than $2 \cdot 10^{-7}$ - the result for $\mu = 128$
can be safely approximated as $1 - e^{-1}$.

\section{Insecurity of two-round keyed Feistel cipher}

Let $x_0 || y_0$ be the first and second $n$-bit halves of the input string.
Let $x_1 || y_1$ be the respective parts after the first Feistel round, and
$x_2 || y_2$ after the second Feistel round. Let $k_1, k_2$ be the two used
keys.

A two-round Feistel cipher with arbitrary round function $F$ and keys $k_1,
k_2$ is not secure, as:

\begin{proof}
	By the definition of $F^*$ we know that:
	\begin{align*}
		x_1 || y_1 & = F_{k_1}^*(x_0 || y_0) = y_0 || (F(k_1, y_0) \oplus x_0) \\
		x_2 || y_2 & = F_{k_2}^*(x_1 || y_2) = (F(k_1, y_0) \oplus x_0) || (F(k_2, F(k_1, y_0) \oplus x_0) \oplus (F(k_1, y_0) \oplus x_0))
	\end{align*}

	Note that $c:= (F(k_1, y_0) \oplus x_0)$ appears multiple times in the expression above:
	\begin{align*}
		x_1 || y_1 & = F_{k_1}^*(x_0 || y_0) = y_0 || c \\
		x_2 || y_2 & = F_{k_2}^*(x_1 || y_2) = c || (F(k_2, c) \oplus c)
	\end{align*}
\end{proof}


\end{document}


