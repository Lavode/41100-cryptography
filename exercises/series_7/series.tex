\documentclass[a4paper]{scrreprt}

% Uncomment to optimize for double-sided printing.
% \KOMAoptions{twoside}

% Set binding correction manually, if known.
% \KOMAoptions{BCOR=2cm}

% Localization options
\usepackage[english]{babel}
\usepackage[T1]{fontenc}
\usepackage[utf8]{inputenc}

% Quotations
\usepackage{dirtytalk}

% Enhanced verbatim sections. We're mainly interested in
% \verbatiminput though.
\usepackage{verbatim}

% PDF-compatible landscape mode.
% Makes PDF viewers show the page rotated by 90°.
\usepackage{pdflscape}

% Advanced tables
\usepackage{tabu}
\usepackage{longtable}

% Fancy tablerules
\usepackage{booktabs}

% Graphics
\usepackage{graphicx}

% Current time
\usepackage[useregional=numeric]{datetime2}

% Float barriers.
% Automatically add a FloatBarrier to each \section
\usepackage[section]{placeins}

% Custom header and footer
\usepackage{fancyhdr}

\usepackage{geometry}
\usepackage{layout}

% Math tools
\usepackage{mathtools}
% Math symbols
\usepackage{amsmath,amsfonts,amssymb}
\usepackage{amsthm}

\DeclarePairedDelimiter\abs{\lvert}{\rvert}

% Indistinguishable operator (three stacked tildes)
\newcommand*{\diffeo}{% 
  \mathrel{\vcenter{\offinterlineskip
  \hbox{$\sim$}\vskip-.35ex\hbox{$\sim$}\vskip-.35ex\hbox{$\sim$}}}}

\pagestyle{plain}
% \fancyhf{}
% \lhead{}
% \lfoot{}
% \rfoot{}
% 
% Source code & highlighting
\usepackage{listings}

% Coloured boxes!
\usepackage[most]{tcolorbox}
\newtcolorbox{library}[2][]{
	enhanced,
	sharp corners,
	coltitle=black, % title colour
	colbacktitle=black!10!white, % title bg colour
	halign title=center, % title align
	toptitle=1mm, % Top/bottom additional spacing for title
	bottomtitle=1mm,
	fonttitle=\ttfamily,
	colback=white, % body bg colour
	fontupper=\ttfamily,
	title=#2,#1
}

\newtcolorbox{boxcomment}[2][]{
	enhanced,
	colframe=white, % frame colour
	colbacktitle=white, % title bg colour
	halign=center, % body align
	colback=white, % body bg colour
	fonttitle=\ttfamily,
	fontupper=\ttfamily,
	title=#2,#1
}

% SI units
\usepackage[binary-units=true]{siunitx}
\DeclareSIUnit\cycles{cycles}

% Convenience commands
\newcommand{\mailsubject}{41100 - Cryptography - Series 7}
\newcommand{\maillink}[1]{\href{mailto:#1?subject=\mailsubject}
                               {#1}}

% Should use this command wherever the print date is mentioned.
\newcommand{\printdate}{\today}

\subject{41100 - Cryptography}
\title{Series 7}

\author{Michael Senn \maillink{michael.senn@students.unibe.ch} - 16-126-880}

\date{\printdate}

% Needs to be the last command in the preamble, for one reason or
% another. 
\usepackage{hyperref}


\begin{document}
\maketitle


\setcounter{chapter}{6}
\chapter{Series 7}

Let $\Sigma := \{0, 1\}$.

\section{PRF using PRG}

Let $F: \Sigma^\lambda \times \Sigma^{2 \lambda} \rightarrow \Sigma^{2
\lambda}$ be a secure PRF, $G: \Sigma^\lambda \rightarrow \Sigma^{2 \lambda}$ a
secure PRG, $F': \Sigma^\lambda \times \Sigma^\lambda \rightarrow \Sigma^{2
\lambda}: x \rightarrow F(k, G(x))$ a candidate PRG.

As $G(x) \in \Sigma^{2\lambda}$ for all input values $x \in \Sigma^\lambda$,
$G(x)$ is guaranteed to be a valid input for $F$. Further as $G$ is injective,
there exists for all $y \in \Sigma^{2 \lambda}$ at most one $x \in
\Sigma^\lambda$ such that $G(x) = y$.

With this in mind, choosing any $x$ in $\Sigma^\lambda$, and using $G(x)$ as
input for $F$ is no different from choosing any $x$ in the image of $G$ and
using it as input for $F$.

As $F$ is secure for all valid inputs, $F'(k, x) = F(k, G(x))$ is therefore
also secure for all choices of $x$, and is hence a secure PRF.

\subsection{Security of $F'$}


\subsection{Insecurity of $F'$ if constructed with $\tilde{G}$} 

Consider the following distinguisher:

\begin{library}{$B$}
	\begin{lstlisting}[mathescape=true]
		$x \leftarrow \{0, 1\}^{\lambda - 1}$
		$x_1 := x || 0$
		$x_2 := x || 1$

		return LOOKUP($x_1$) == LOOKUP($x_2$)
	\end{lstlisting}
\end{library}

By the nature of its construction:
\[
	P[B \diamond L^{F'}_{\text{PRG-real}} \rightarrow 1] = 1
\]

And:
\[
	P[B \diamond L^{F'}_{\text{PRG-rand}} \rightarrow 1] = 2^{-2\lambda}
\]

Hence:
\[
	\lim_{\lambda \to \infty} Bias(B) = \lim_{\lambda \to \infty} 1 - 2^{-2\lambda} = 1
\]

As $\lim_{\lambda \to \infty} Bias(B) \neq 0$ it is non-negligible, so F' is
not secure.

\section{Pseudo-random permutations}

Let $\alpha: \Sigma^\mu \rightarrow \Sigma^\mu$ be a random permutation on
$\Sigma^\mu$. Let $F$ be a secure PRP, $k \in \Sigma^\lambda$ its key.

\subsection{Probability of conflict}


A PRP can produce at most $2^\lambda$ distinct permutations - one for each
possible $\lambda$-bit key - as its output has to be deterministic for any
given $x$ and $k$.

Further there are $\mu!$ permutations on a collection of $\mu$ elements, as
there are $\mu$ choices for the first element, $\mu - 1$ for the second and so
on.

As such the chance of a random permutation on $\Sigma^\mu$ being one of the
permutations produced by $F$ is:

\[
	\frac{2^\lambda}{\mu!}
\]

\subsection{$\lambda = \mu = 128$}

With $\lambda = \mu = 128$:

\[
	\frac{2^{128}}{128!} = 8.8 \cdot 10^{-178} \approx 0
\]

This being extremely close to $0$ illustrates the sheer number of total
permutations compared to the number of permutations which can be generated by a
PRP.

\section{Insecurity of two-round keyed Feistel cipher}

Let $x_0 || y_0$ be the first and second $n$-bit halves of the input string.
Let $x_1 || y_1$ be the respective parts after the first Feistel round, and
$x_2 || y_2$ after the second Feistel round. Let $k_1, k_2$ be the two used
keys.

By the definition of $F^*$ we know that:
\begin{align*}
	x_1 || y_1 & = y_0 & || & F(k_1, y_0) \oplus x_0 \\
	x_2 || y_2 & = F(k_1, y_0) \oplus x_0 & || & F(k_2, F(k_1, y_0) \oplus x_0) \oplus x_1
\end{align*}

A two-round Feistel cipher is not secure against a chosen-plaintext attack:
\begin{proof}
	Let $i_1 := x_0 || y_0$, $i_2 := x'_0 || y_0$ be two inputs with equal
	right half. Let $x_2$, $x'_2$ be the first halves of the outputs of the
	two inputs, encrypted with a two-round Feistel cipher. As per above:
	\begin{align*}
		x_2 & = F(k_1, y_0) \oplus x_0
		x'_2 & = F(k_1, y_0) \oplus x'_0
	\end{align*}

	Note that the two differ only in one of the two operands of the XOR
	operation - an operand which, in the case of a chosen-plaintext attack
	- we control.

	As such we can reconstruct it as follows:
	\begin{align*}
		c = x_2 \oplus x_0 \\
		c' = x'_2 \oplus x'_0
	\end{align*}

	In the case of a two-round Feistel cipher, $P[c = c'] = 1$. In the case
	of a truly random permutation on $2\lambda$ bit inputs, $P[c = c'] =
	\frac{1}{2^\lambda}$. As such this allows us to distinguish the two
	with an advantage of $1 - 2^{-\lambda}$, which is non-negligible.
\end{proof}

\end{document}


